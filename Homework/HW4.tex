%%%%%%%%%%%%%%%%%%%%%%%%%%%%%%%%%%%%%%%%%%%%%%%%%%%%%%%%%%%%%%%%%%%%%%%%%%%%%%%%%%%%%%%%%%%%%%%%%%%%%%%%%%%%%%%%%%%%%%%%%%%%%%%%%%%%%%%%%%%%%%%%%%%%%%%%%%%%%%%%%%%
% Written By Michael Brodskiy
% Class: Quantum Mechanics
% Professor: G. Fiete
%%%%%%%%%%%%%%%%%%%%%%%%%%%%%%%%%%%%%%%%%%%%%%%%%%%%%%%%%%%%%%%%%%%%%%%%%%%%%%%%%%%%%%%%%%%%%%%%%%%%%%%%%%%%%%%%%%%%%%%%%%%%%%%%%%%%%%%%%%%%%%%%%%%%%%%%%%%%%%%%%%%

\documentclass[12pt]{article} 
\usepackage{alphalph}
\usepackage[utf8]{inputenc}
\usepackage[russian,english]{babel}
\usepackage{titling}
\usepackage{amsmath}
\usepackage{float}
\usepackage{graphicx}
\usepackage{enumitem}
\usepackage{amssymb}
\usepackage[super]{nth}
\usepackage{everysel}
\usepackage{ragged2e}
\usepackage{geometry}
\usepackage{multicol}
\usepackage{fancyhdr}
\usepackage{cancel}
\usepackage{siunitx}
\usepackage{physics}
\usepackage{tikz}
\usepackage{mathdots}
\usepackage{yhmath}
\usepackage{cancel}
\usepackage{color}
\usepackage{array}
\usepackage{multirow}
\usepackage{gensymb}
\usepackage{tabularx}
\usepackage{extarrows}
\usepackage{booktabs}
\usepackage{lastpage}
\usetikzlibrary{fadings}
\usetikzlibrary{patterns}
\usetikzlibrary{shadows.blur}
\usetikzlibrary{shapes}

\geometry{top=1.0in,bottom=1.0in,left=1.0in,right=1.0in}
\newcommand{\subtitle}[1]{%
  \posttitle{%
    \par\end{center}
    \begin{center}\large#1\end{center}
    \vskip0.5em}%

}
\usepackage{hyperref}
\hypersetup{
colorlinks=true,
linkcolor=blue,
filecolor=magenta,      
urlcolor=blue,
citecolor=blue,
}


\title{Homework 4}
\date{\today}
\author{Michael Brodskiy\\ \small Professor: G. Fiete}

\begin{document}

\maketitle

\begin{enumerate}

  \item

    \begin{enumerate}

      \item Given the measurement, we know that:

        $$\ket{\psi(0)}=\ket{+}_x$$

        Or, alternatively:

        $$\boxed{\ket{\psi(0)}=\frac{1}{\sqrt{2}}(\ket{+}+\ket{-})}$$

      \item First and foremost, we may write the Hamiltonian as:

        $$\vec{H}=\mu\vec{B}$$

        Furthermore, we know that a quantum state will evolve with time according to:

        $$\ket{\psi(t)}=\ket{\psi(0)}e^{-\frac{iEt}{\hbar}}$$

        With $E$ representing the energy of the particle. Given that the particle is in a $\hat{\bold{z}}$ only orientation, we may write the Hamiltonian as:

        $$\vec{H}=\mu B_o S_z$$

        Using the Larmor precession frequency, $\omega_o$, we find the eigenvalues of this state to be:

        $$E_{\pm}=\lambda_{\pm}=\pm\frac{\omega_o\hbar}{2}$$

        Thusly, combining the time-evolving state formula along with the above energy eigenvalues allows us to obtain the time-evolving state  at $t=T$ as:

        $$\boxed{\ket{\psi(T)}=\frac{1}{\sqrt{2}}\left[e^{-\frac{i\omega_o T}{2}}\ket{+}+e^{\frac{i\omega_o T}{2}}\ket{-}\right]}$$

      \item First, we can take the $\ket{\psi(T)}$ as the new $\ket{\psi(0)}$, such that:

        $$\ket{\psi(0)}=\frac{1}{\sqrt{2}}\left[e^{-\frac{i\omega_o T}{2}}\ket{+}+e^{\frac{i\omega_o T}{2}}\ket{-}\right]$$

        Given that we want this aligned with the $y$ axis due to the new basis, we may rewrite the above as:

        $$\ket{\psi(0)}=\frac{1}{2}\left[\left(e^{-\frac{i\omega_o T}{2}}-ie^{\frac{i\omega_o T}{2}}\right)\ket{+}_y+\left(e^{-\frac{i\omega_o T}{2}}+ie^{\frac{i\omega_o T}{2}}\right)\ket{-}_y\right]$$

        We can then substitute into our formula for time evolution to get:

        $$\ket{\psi(t)}=\frac{1}{2}\left[\left(e^{-\frac{i\omega_o T}{2}}-ie^{\frac{i\omega_o T}{2}}\right)e^{-\frac{iE_+t}{\hbar}}\ket{+}_y+\left(e^{-\frac{i\omega_o T}{2}}+ie^{\frac{i\omega_o T}{2}}\right)e^{-\frac{iE_-t}{\hbar}}\ket{-}_y\right]$$

        Note that, because of the similar form of the magnetic field, the eigenvalues take the same scalar values, just different orientations, for the Hamiltonian. Thus, we get:

        $$\ket{\psi(t)}=\frac{1}{2}\left[\left(e^{-\frac{i\omega_o T}{2}}-ie^{\frac{i\omega_o T}{2}}\right)e^{-\frac{i\omega_ot}{2}}\ket{+}_y+\left(e^{-\frac{i\omega_o T}{2}}+ie^{\frac{i\omega_o T}{2}}\right)e^{\frac{i\omega_ot}{2}}\ket{-}_y\right]$$

        We then take $t\to T$ to get the state at time $T$:

        $$\ket{\psi(T)}=\frac{1}{2}\left[\left(e^{-i\omega_o T}-i\right)\ket{+}_y+\left(1+ie^{i\omega_o T}\right)\ket{-}_y\right]$$

        We then use our probability formula:

        $$P_a=|\bra{a}\ket{\psi(t)}|^2$$

        We can take $a\to\ket{+}_x$ and $t\to T$ to write:

        $$P_{+x}=|\,\,_x\bra{+}\ket{\psi(T)}|^2$$

        We expand and use matrix notation to get:

        $$P_{+x}=\Big|\frac{1}{\sqrt{2}} (1\quad1 )\frac{1}{2}\left[\left(e^{-i\omega_o T}-i\right)\frac{1}{\sqrt{2}}\left( \begin{matrix}1\\i \end{matrix}\right)+\left(1+ie^{i\omega_o T}\right)\frac{1}{\sqrt{2}}\left( \begin{matrix} 1\\-i\end{matrix} \right)\right]\Big|^2$$

        And now, we simply evaluate:

        $$P_{+x}=\frac{1}{16}\Big|(1\quad1 )\left[\left(e^{-i\omega_o T}-i\right)\left( \begin{matrix}1\\i \end{matrix}\right)+\left(1+ie^{i\omega_o T}\right)\left( \begin{matrix} 1\\-i\end{matrix} \right)\right]\Big|^2$$
        $$P_{+x}=\frac{1}{16}\Big|\left(e^{-i\omega_o T}-i\right)(1\quad 1)\left( \begin{matrix}1\\i \end{matrix}\right)+\left(1+ie^{i\omega_o T}\right)(1\quad1)\left( \begin{matrix} 1\\-i\end{matrix} \right)\Big|^2$$
        $$P_{+x}=\frac{1}{16}\Big|\left(e^{-i\omega_o T}-i\right)(1+i)+\left(1+ie^{i\omega_o T}\right)(1-i)\Big|^2$$
        $$P_{+x}=\frac{1}{16}\Big|\left(e^{-i\omega_o T}-i+ie^{-i\omega_o T}+1\right)+\left(1+ie^{i\omega_o T}-i+e^{i\omega_o T}\right)\Big|^2$$

        Per our trigonometric identities, this gives us:

        $$P_{+x}=\frac{1}{4}\Big|i\cos(\omega_o T)+\cos(\omega_o T)-i+1\Big|^2$$

        We then take the magnitude and square to get:

        $$P_{+x}=\frac{1}{4}(1+\cos^2(\omega_o T))$$

        Using our double-angle formula, we get:

        $$\boxed{P_{+x}=\frac{3}{4}+\frac{\cos(2\omega_oT)}{4}}$$

    \end{enumerate}

  \item With the given Hamiltonian, we may observe:

    $$\lambda=E_1,E_2$$

    And the corresponding eigenstates are:

    $$\lambda=E_1\to\left[ \begin{matrix} 1\\0\end{matrix} \right]\quad\text{ and }\quad \lambda=E_2\to\left[ \begin{matrix} 0\\1\end{matrix} \right]$$

    Similarly, we may see that the eigenvalues for $\hat{A}$ are:

    $$\lambda=\pm a$$

    And the corresponding eigenstates are:

    $$\lambda=a\to\frac{1}{\sqrt{2}}\left[ \begin{matrix} 1\\1\end{matrix} \right]\quad\text{ and }\quad \lambda=-a\to\frac{1}{\sqrt{2}}\left[ \begin{matrix} 1\\-1\end{matrix} \right]$$

    Thus, we may conclude:

    $$\ket{\psi(0)}=\frac{1}{\sqrt{2}}\left[ \begin{matrix} 1\\1\end{matrix} \right]$$

    In terms of the Hamiltonian's eigenstates, we may observe:

    $$\ket{\psi(0)}=\frac{1}{\sqrt{2}}\left(\left[ \begin{matrix} 1\\0\end{matrix} \right]+\left[ \begin{matrix} 0\\1\end{matrix} \right]\right)$$

    We know that the time-evolution of the state may be written as:

    $$\ket{\psi(t)}=\ket{\psi(0)}e^{-\frac{iEt}{\hbar}}$$

    Thus, we get:

    $$\ket{\psi(t)}=\frac{1}{\sqrt{2}}\left[ \begin{matrix} e^{-iE_2t/\hbar}\\ e^{-iE_1t/\hbar} \end{matrix}\right]$$

    We want to find the expectation value of $\hat{A}$, so we write:

    $$<\hat{A}>=\bra{\psi(t)|\hat{A}}\ket{\psi(t)}$$

    Writing this in matrix form, we get:

    $$<\hat{A}>=\frac{1}{2}\left[ e^{iE_2t/\hbar}\quad e^{iE_1t/\hbar} \right]\left[ \begin{matrix} 0 & a\\a & 0\end{matrix} \right]\left[ \begin{matrix} e^{-iE_2t/\hbar}\\ e^{-iE_1t/\hbar} \end{matrix}\right]$$

    Now, we evaluate:

    $$<\hat{A}>=\frac{1}{2}\left[ ae^{iE_1t/\hbar}\quad ae^{iE_2t/\hbar} \right]\left[ \begin{matrix} e^{-iE_2t/\hbar}\\ e^{-iE_1t/\hbar} \end{matrix}\right]$$
    $$<\hat{A}>=\frac{1}{2}\left[ ae^{i(E_1-E_2)t/\hbar}+ae^{i(E_2-E_1)t/\hbar} \right]$$

    We use our trigonometric identities to write:

    $$<\hat{A}>=a\cos\left( \frac{E_2-E_1}{\hbar}t \right)$$

    As such, we may see:

    $$\boxed{\omega=\frac{E_2-E_1}{\hbar}\Rightarrow f=\frac{E_2-E_1}{h}}$$

  \item

    \begin{enumerate}

      \item We determine this by finding the eigenvalues and corresponding eigenstates of the Hamiltonian. This gives us:

        $$\Big|\left[ \begin{matrix} 2-\lambda & 1\\ 1 & 2-\lambda\end{matrix} \right]\Big|=0$$

        We solve:

        $$(2-\lambda)^2-1=0$$
        $$4-4\lambda+\lambda^2-1=0$$
        $$\lambda^2-4\lambda+3=0$$
        $$(\lambda-3)(\lambda-1)=0$$

        Thus, we get (note we reintroduce the factor of $E_o$):

        $$\boxed{\lambda=1E_o,3E_o}$$

        As such, we see that we can measure either 1 or 3 for the energy. The corresponding eigenstates become:

        $$\boxed{\lambda=1\to\frac{1}{\sqrt{2}}\left[ \begin{matrix} 1\\-1\end{matrix} \right]\quad\text{ and }\lambda=3\to\frac{1}{\sqrt{2}}\left[ \begin{matrix} 1\\1\end{matrix} \right]}$$

        We apply our probability formula to get:

        $$P_{1}=|\bra{1}\ket{\hat{A}}|^2$$

        Writing the eigenstate for $\ket{1}$ in terms of the $S_z$ basis gives us:

        $$\ket{1}=\frac{1}{\sqrt{2}}[\ket{a_1}-\ket{a_2}]$$

        Similarly, we get:

        $$\ket{3}=\frac{1}{\sqrt{2}}[\ket{a_1}+\ket{a_2}]$$

        Using the two states above, we may write:

        $$\ket{a_1}=\frac{1}{\sqrt{2}}[\ket{1}+\ket{3}]$$
        $$\ket{a_2}=\frac{1}{\sqrt{2}}[\ket{1}-\ket{3}]$$

        We normalize the initial state by finding $C=5$ to get:

        $$\ket{\psi(0)}=\frac{3}{5}\ket{a_1}+\frac{4}{5}\ket{a_2}$$

        We then write the probability in matrix form to get:

        $$P_1=\frac{1}{2}\Big|[\ket{a_1}-\ket{a_2}]\left[ \frac{3}{5}\ket{a_1}+\frac{4}{5}\ket{a_2} \right]\Big|^2$$

        We solve to get:

        $$P_1=\frac{1}{2}\Big|-\frac{1}{5}\Big|^2$$
        $$\boxed{P_1=\frac{1}{50}}$$

        Then we find:

        $$P_3=\frac{1}{2}\Big|[\ket{a_1}+\ket{a_2}]\left[ \frac{3}{5}\ket{a_1}+\frac{4}{5}\ket{a_2} \right]\Big|^2$$
        $$P_3=\frac{1}{2}\Big|\frac{7}{5}\Big|^2$$
        $$\boxed{P_3=\frac{49}{50}}$$

      \item Next, we calculate the expectation value:

        $$<\hat{A}>=\bra{\psi(t)\Big|\hat{A}}\ket{\psi(t)}$$

        Writing $\psi(0)$ in terms of the Hamiltonian eigenstates gives us:

        $$\ket{\psi(0)}=-\frac{1}{5\sqrt{2}}\ket{1}+\frac{7}{5\sqrt{2}}\ket{3}$$

        Applying this to the time-evolution formula, we get:

        $$\ket{\psi(t)}=-\frac{1}{5\sqrt{2}}\ket{1}e^{-iE_ot/\hbar}+\frac{7}{5\sqrt{2}}\ket{3}e^{-3iE_ot/\hbar}$$

        We then rewrite in terms of $a_1$ and $a_2$ eigenstates to get:

        $$\ket{\psi(t)}=-\frac{1}{10}\left[e^{-iE_ot/\hbar}-2e^{-3iE_ot/\hbar}\right]\ket{a_1}+\frac{1}{10}\left[e^{-iE_ot/\hbar}+2e^{-3iE_ot/\hbar}\right]\ket{a_2}$$

        By the definitions of operators, we know:

        $$\hat{A}\ket{a_i}=a_i\ket{a_i}$$

        So we obtain:

        $$<\hat{A}>=a_1\left[ -\frac{1}{10}e^{ie_ot/\hbar}+\frac{7}{10}e^{3ie_ot/\hbar} \right]\left[ -\frac{1}{10}e^{-ie_ot/\hbar}+\frac{7}{10}e^{-3ie_ot/\hbar} \right]+$$$$a_2\left[ \frac{1}{10}e^{ie_ot/\hbar}+\frac{7}{10}e^{3ie_ot/\hbar} \right]\left[ \frac{1}{10}e^{-ie_ot/\hbar}+\frac{7}{10}e^{-3ie_ot/\hbar} \right]$$

        From here, we continue to simplify:

        $$<\hat{A}>=a_1\left[\frac{1}{2}-\frac{7}{100}e^{-2iE_ot/\hbar}-\frac{7}{100}e^{2iE_ot/\hbar}\right]+a_2\left[\frac{1}{2}+\frac{7}{100}e^{-2iE_ot/\hbar}+\frac{7}{100}e^{2iE_ot/\hbar}\right]$$
        $$<\hat{A}>=\frac{(a_1+a_2)}{2}-\frac{7a_1}{50}\cos\left( \frac{2E_ot}{\hbar} \right)+\frac{7a_2}{50}\cos\left( \frac{2E_ot}{\hbar}\right) \right]$$

        As such, we can finally obtain:

        $$\boxed{<\hat{A}>=\frac{(a_1+a_2)}{2}+\frac{7(a_2-a_1)}{50}\cos\left( \frac{2E_ot}{\hbar} \right)}$$
        
    \end{enumerate}

\end{enumerate}

\end{document}

