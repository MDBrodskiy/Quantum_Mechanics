%%%%%%%%%%%%%%%%%%%%%%%%%%%%%%%%%%%%%%%%%%%%%%%%%%%%%%%%%%%%%%%%%%%%%%%%%%%%%%%%%%%%%%%%%%%%%%%%%%%%%%%%%%%%%%%%%%%%%%%%%%%%%%%%%%%%%%%%%%%%%%%%%%%%%%%%%%%%%%%%%%%
% Written By Michael Brodskiy
% Class: Quantum Mechanics
% Professor: G. Fiete
%%%%%%%%%%%%%%%%%%%%%%%%%%%%%%%%%%%%%%%%%%%%%%%%%%%%%%%%%%%%%%%%%%%%%%%%%%%%%%%%%%%%%%%%%%%%%%%%%%%%%%%%%%%%%%%%%%%%%%%%%%%%%%%%%%%%%%%%%%%%%%%%%%%%%%%%%%%%%%%%%%%

\documentclass[12pt]{article} 
\usepackage{alphalph}
\usepackage[utf8]{inputenc}
\usepackage[russian,english]{babel}
\usepackage{titling}
\usepackage{amsmath}
\usepackage{float}
\usepackage{graphicx}
\usepackage{enumitem}
\usepackage{amssymb}
\usepackage[super]{nth}
\usepackage{everysel}
\usepackage{ragged2e}
\usepackage{geometry}
\usepackage{multicol}
\usepackage{fancyhdr}
\usepackage{cancel}
\usepackage{siunitx}
\usepackage{physics}
\usepackage{tikz}
\usepackage{mathdots}
\usepackage{yhmath}
\usepackage{cancel}
\usepackage{color}
\usepackage{array}
\usepackage{multirow}
\usepackage{gensymb}
\usepackage{tabularx}
\usepackage{extarrows}
\usepackage{booktabs}
\usepackage{lastpage}
\usetikzlibrary{fadings}
\usetikzlibrary{patterns}
\usetikzlibrary{shadows.blur}
\usetikzlibrary{shapes}

\geometry{top=1.0in,bottom=1.0in,left=1.0in,right=1.0in}
\newcommand{\subtitle}[1]{%
  \posttitle{%
    \par\end{center}
    \begin{center}\large#1\end{center}
    \vskip0.5em}%

}
\usepackage{hyperref}
\hypersetup{
colorlinks=true,
linkcolor=blue,
filecolor=magenta,      
urlcolor=blue,
citecolor=blue,
}


\title{Homework 3}
\date{\today}
\author{Michael Brodskiy\\ \small Professor: G. Fiete}

\begin{document}

\maketitle

\begin{enumerate}

  \item

    \begin{enumerate}

      \item We know that, for a spin-1 system, the spin can be:

        $$\boxed{S_z=\hbar,0,-\hbar}$$

        Given the quantum state in the problem, we can find the probabilities of each as:

        $$P_1=\left( \frac{1}{\sqrt{30}} \right)^2$$
        $$P_0=\left( \frac{2}{\sqrt{30}} \right)^2$$
        $$P_{-1}=\left( \frac{|5i|}{\sqrt{30}} \right)^2$$

        This gives us:

        $$\boxed{\left\{\begin{array}{ll} P_1&= 1/30\\P_0&= 2/15\\P_{-1}&= 5/6\end{array}}$$

          We can calculate the expectation value of $S_z$ by using:

          $$\bra{\psi|S_z}\ket{\psi}$$

          In matrix notation, this gives us:

          $$<S_z>=\left[ \frac{\hbar}{30} \right](1\quad 2\quad -5i)\left( \begin{matrix} 1 & 0 & 0\\ 0 & 0 & 0\\ 0 & 0 & -1\end{matrix} \right)\left( \begin{matrix}1\\2\\5i\end{matrix} \right)$$

          We proceed to evaluate:

          $$<S_z>=\left[ \frac{\hbar}{30} \right](1\quad 0\quad 5i)\left( \begin{matrix}1\\2\\5i\end{matrix} \right)$$
          $$<S_z>=\left[ \frac{\hbar}{30} \right](1-(25))$$
          $$\boxed{<S_z>= -\frac{24\hbar}{30}=-\frac{12\hbar}{15}}$$

        \item Similar to the expectation value of $S_z$ in part (a), we apply the matrix notation of $S_x$ to write:

          $$<S_x>=\left[ \frac{\hbar}{30\sqrt{2}} \right](1\quad 2\quad -5i)\left( \begin{matrix} 0 & 1 & 0\\ 1 & 0 & 1\\ 0 & 1 & 0\end{matrix} \right)\left( \begin{matrix}1\\2\\5i\end{matrix} \right)$$
          $$<S_x>=\left[ \frac{\hbar}{30\sqrt{2}} \right](2\quad 1-5i\quad 2)\left( \begin{matrix}1\\2\\5i\end{matrix} \right)$$
          $$<S_x>=\left[ \frac{\hbar}{30\sqrt{2}} \right](4+10i-10i)$$
          $$\boxed{<S_x>= \frac{4\hbar}{30\sqrt{2}}=\frac{2\hbar}{15\sqrt{2}}}$$

    \end{enumerate}

  \item

  \item

    \begin{enumerate}

      \item We know that $\hat{A}$ and $\hat{B}$ commute if $\hat{A}\hat{B}=\hat{B}\hat{A}$. Thus, we begin by calculating $\hat{A}\hat{B}$:

        $$\hat{A}\hat{B}=\left( \begin{matrix} a_1 & 0 & 0\\ 0 & a_2 & 0\\ 0 & 0 & a_3\end{matrix} \right)\left( \begin{matrix} b_1 & 0 & 0\\ 0 & 0 & b_2\\ 0 & b_2 & 0\end{matrix} \right)$$

        This gives us:

        $$\hat{A}\hat{B}=\left( \begin{matrix} a_1b_1 & 0 & 0\\ 0 & 0 & a_2b_2\\ 0 & a_3b_2 & 0\end{matrix} \right)$$

        Now, we calculate $\hat{B}\hat{A}$:

        $$\hat{B}\hat{A}=\left( \begin{matrix} b_1 & 0 & 0\\ 0 & 0 & b_2\\ 0 & b_2 & 0\end{matrix} \right)\left( \begin{matrix} a_1 & 0 & 0\\ 0 & a_2 & 0\\ 0 & 0 & a_3\end{matrix} \right)$$

        This gives us:

        $$\hat{B}\hat{A}=\left( \begin{matrix} b_1a_1 & 0 & 0\\ 0 & 0 & b_2a_3\\ 0 & b_2a_2&0\end{matrix} \right)$$

        We may see that $\hat{A}\hat{B}\neq\hat{B}\hat{A}$ and, therefore, the operators \underline{do not commute}

      \item We begin working with $\hat{A}$:

        $$|\hat{A}-\lambda\mathbb{1}|=0$$

        This gives us:

        $$(a_1-\lambda)(a_2-\lambda)(a_3-\lambda)=0$$

        Thus, we see that, since this is a diagonal matrix, we get $\boxed{\lambda=a_1, a_2,\text{ and }a_3}$ as the eigenvalues. Thus, we may observe that the normalized eigenvectors are:

        $$\boxed{\left\{ \left[ \begin{matrix} 1\\0\\0\end{matrix} \right], \left[ \begin{matrix}0\\1\\0\end{matrix} \right],\left[ \begin{matrix} 0\\0\\1\end{matrix} \right] \right\}}$$

        We then proceed to do the same with $\hat{B}$:

        $$\Big|\left[ \begin{matrix} b_1-\lambda & 0 & 0\\ 0 & -\lambda & b_2\\0 & b_2 & -\lambda\end{matrix} \right]\Big|=0$$

        This gives us:

        $$(b_1-\lambda)(\lambda^2-b_2^2)=0$$

        We can thus see that the eigenvalues are:

        $$\boxed{\lambda=b_1,\pm b_2}$$

        From here, we can observe that the normalized eigenvectors become:

        $$\boxed{\left[ \begin{matrix} 1 \\ 0 \\ 0 \end{matrix}\right],\frac{1}{\sqrt{2}}\left[ \begin{matrix} 0\\1\\1\end{matrix} \right],\frac{1}{\sqrt{2}}\left[ \begin{matrix}0\\-1\\1\end{matrix} \right]}$$

      \item Let us assume that the basis eigenvectors are:

        $$\ket{1}= \left[ \begin{matrix} 1\\0\\0\end{matrix} \right],\quad\ket{2}=\left[ \begin{matrix}0\\1\\0\end{matrix} \right],\ket{3}=\left[ \begin{matrix} 0\\0\\1\end{matrix} \right]$$

        We calculate the measure of $\hat{B}$ by writing:

        $$\hat{B}\ket{2}=\left( \begin{matrix} b_1 & 0 & 0\\ 0 & 0 & b_2\\ 0 & b_2 & 0\end{matrix} \right)\left( \begin{matrix}0\\1\\0\end{matrix} \right)$$
        $$\hat{B}\ket{2}=\left( \begin{matrix}0\\0\\b_2\end{matrix} \right)$$

        Rewriting this in terms of the basis vectors of $\hat{B}$, we may find:

        $$\hat{B}\ket{2}=\frac{b_2}{2}\left(\left[ \begin{matrix} 0\\1\\1\end{matrix} \right]+\left[ \begin{matrix}0\\-1\\1\end{matrix} \right]  \right)$$
        $$\boxed{\hat{B}\ket{2}=\frac{b_2}{\sqrt{2}}\left(\ket{B_2}+\ket{B_3}  \right)}$$

        Since we may see that the two occur with equal probabilities, we may say that, for the operator $\hat{B}$, the probabilities are $P_{B_2}=P_{B_3}=.5$

        On the other hand, we may find $a$ to be:

        $$\hat{A}\left( \begin{matrix} 0\\0\\b_2\end{matrix} \right)=\left( \begin{matrix} a_1 & 0 & 0\\ 0 & a_2 & 0\\ 0 & 0 & a_3\end{matrix} \right)\left( \begin{matrix}0\\0\\b_2\end{matrix} \right)$$
        $$\hat{A}\left( \begin{matrix} 0\\0\\b_2\end{matrix} \right)=\left( \begin{matrix} 0\\0\\b_2a_3\end{matrix}\right)$$

        We may see that this is equivalent to an integer multiple of the eigenvector $\ket{A_3}$ of operator $\hat{A}$. Thus, we may find that:

        $$\hat{A}\hat{B}\ket{2}=b_2a_3\ket{A_3}$$

        And, therefore, \underline{the measurement produces $a_3$ with a probability of $1$}

      \item Note that the result from part (a) indicates that we can not know the measurements of $A$ and $B$ at the same time with certainty. This is confirmed by the results from part (c), as we find that measuring $\hat{B}\ket{2}$ before $\hat{A}$ has no effect on the result of $\hat{A}$

    \end{enumerate}

\end{enumerate}

\end{document}

