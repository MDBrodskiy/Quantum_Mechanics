%%%%%%%%%%%%%%%%%%%%%%%%%%%%%%%%%%%%%%%%%%%%%%%%%%%%%%%%%%%%%%%%%%%%%%%%%%%%%%%%%%%%%%%%%%%%%%%%%%%%%%%%%%%%%%%%%%%%%%%%%%%%%%%%%%%%%%%%%%%%%%%%%%%%%%%%%%%%%%%%%%%
% Written By Michael Brodskiy
% Class: Quantum Mechanics
% Professor: G. Fiete
%%%%%%%%%%%%%%%%%%%%%%%%%%%%%%%%%%%%%%%%%%%%%%%%%%%%%%%%%%%%%%%%%%%%%%%%%%%%%%%%%%%%%%%%%%%%%%%%%%%%%%%%%%%%%%%%%%%%%%%%%%%%%%%%%%%%%%%%%%%%%%%%%%%%%%%%%%%%%%%%%%%

\include{Includes.tex}

\title{Homework 8}
\date{\today}
\author{Michael Brodskiy\\ \small Professor: G. Fiete}

\begin{document}

\maketitle

\begin{enumerate}

  \item First and foremost, we have the wave function as:

    $$\psi_{321}(r,\theta,\phi)=-\frac{\sqrt{3}}{27\sqrt{\pi}}\sqrt[3]{\frac{Z}{3a_o}}\left( \frac{Zr}{a_o} \right)^2e^{-Zr/3a_o}\sin(\theta)\cos(\theta)e^{i\phi}$$

    We can apply the differential form $L_z$ to get:

    $$L_z\psi_{321}(r,\theta,\phi)=-i\hbar\frac{\partial}{\partial\phi}\left[\frac{\sqrt{3}}{27\sqrt{\pi}}\sqrt[3]{\frac{Z}{3a_o}}\left( \frac{Zr}{a_o} \right)^2e^{-Zr/3a_o}\sin(\theta)\cos(\theta)e^{i\phi}\right]$$
    $$L_z\psi_{321}(r,\theta,\phi)=-i\hbar\frac{\sqrt{3}}{27\sqrt{\pi}}\sqrt[3]{\frac{Z}{3a_o}}\left( \frac{Zr}{a_o} \right)^2e^{-Zr/3a_o}\sin(\theta)\cos(\theta)\frac{\partial}{\partial\phi}\left[e^{i\phi}\right]$$
    $$L_z\psi_{321}(r,\theta,\phi)=\hbar\frac{\sqrt{3}}{27\sqrt{\pi}}\sqrt[3]{\frac{Z}{3a_o}}\left( \frac{Zr}{a_o} \right)^2e^{-Zr/3a_o}\sin(\theta)\cos(\theta)e^{i\phi}$$

    We may observe that this gives us $L_z\psi_{321}=\hbar\psi_{321}$, and, therefore, $\psi_{321}$ is an eigenstate of $L_z$ with eigenvalue $\hbar$ (as expected with $m=1$). We now proceed to check $\vec{L}^2$:

    $$\vec{L}^2\psi_{321}(r,\theta,\phi)=-\hbar^2\left[\frac{1}{\sin(\theta)}\frac{\partial}{\partial\theta}\left( \sin(\theta)\frac{\partial}{\partial\theta} \right)+\frac{1}{\sin^2(\theta)}\frac{\partial^2}{\partial\phi^2}\right]\psi_{321}(r,\theta,\phi)$$

    We pull out $r$-dependent terms as constants (let us express these as $\bold{R}$), which gives us:

    $$\vec{L}^2\psi_{321}(r,\theta,\phi)=-\hbar^2R\left[ \frac{1}{\sin(\theta)}\frac{\partial}{\partial\theta}\left[ \sin(\theta)\frac{\partial}{\partial\theta} \sin(\theta)\cos(\theta) \right]e^{i\phi} +\cot(\theta)\frac{\partial^2}{\partial\phi^2} e^{i\phi}\right]$$
    $$\vec{L}^2\psi_{321}(r,\theta,\phi)=-\hbar^2Re^{i\phi}\left[ \frac{1}{\sin(\theta)}\frac{\partial}{\partial\theta}\left[ \sin(\theta)\cos^2(\theta)-\sin^3(\theta) \right]-\cot(\theta)\right]$$
    $$\vec{L}^2\psi_{321}(r,\theta,\phi)=-\hbar^2Re^{i\phi}\left[ \frac{1}{\sin(\theta)}\left[ \cos^3(\theta)-5\cos(\theta)\sin^2(\theta) \right]-\cot(\theta)\right]$$
    $$\vec{L}^2\psi_{321}(r,\theta,\phi)=-\hbar^2Re^{i\phi}\left[ \cos^2(\theta)\cot(\theta)-5\cos(\theta)\sin(\theta)-\cot(\theta)\right]$$

    Using trigonometric identities, we may simplify to:

    $$\vec{L}^2\psi_{321}(r,\theta,\phi)=-\hbar^2Re^{i\phi}\left[ -6\cos(\theta)\sin(\theta)\right]$$

    And finally:

    $$\vec{L}^2\psi_{321}(r,\theta,\phi)=6\hbar^2Re^{i\phi}\left[ \cos(\theta)\sin(\theta)\right]$$
    
    We may see that this indicates that $\psi_{321}$ is, indeed, an eigenfunction of $\vec{L}^2$, with eigenvalue $6\hbar^2$ (which would be expected for $l=2$, since $[2(2+1)]$ gives us $6$). Finally, we check the Hamiltonian, which can be expressed as:

    $$H=-\frac{\hbar^2}{2\mu}\left[ \frac{1}{r^2}\frac{\partial}{\partial r}\left( r^2\frac{\partial}{\partial r} \right)+\frac{1}{r^2\sin(\theta)}\frac{\partial}{\partial \theta}\left( \sin(\theta)\frac{\partial}{\partial\theta}+\frac{1}{r^2\sin(\theta)}\frac{\partial^2}{\partial\phi^2} \right) \right] + V(r)$$

    Before we evaluate, we can simplify this to:

    $$H=-\frac{\hbar^2}{2\mu}\left[ \frac{1}{r^2}\frac{\partial}{\partial r}\left( r^2\frac{\partial}{\partial r}\right) -\frac{\vec{L}^2}{r^2\hbar^2}\right]-\frac{Ze^2}{4\pi\varepsilon_or}$$

    We take $\mu\to m_e$ to give us:

  \item

    \begin{enumerate}

      \item 

      \item 

      \item 

      \item 

    \end{enumerate}

\end{enumerate}

\end{document}

