%%%%%%%%%%%%%%%%%%%%%%%%%%%%%%%%%%%%%%%%%%%%%%%%%%%%%%%%%%%%%%%%%%%%%%%%%%%%%%%%%%%%%%%%%%%%%%%%%%%%%%%%%%%%%%%%%%%%%%%%%%%%%%%%%%%%%%%%%%%%%%%%%%%%%%%%%%%%%%%%%%%
% Written By Michael Brodskiy
% Class: Quantum Mechanics
% Professor: G. Fiete
%%%%%%%%%%%%%%%%%%%%%%%%%%%%%%%%%%%%%%%%%%%%%%%%%%%%%%%%%%%%%%%%%%%%%%%%%%%%%%%%%%%%%%%%%%%%%%%%%%%%%%%%%%%%%%%%%%%%%%%%%%%%%%%%%%%%%%%%%%%%%%%%%%%%%%%%%%%%%%%%%%%

\documentclass[12pt]{article} 
\usepackage{alphalph}
\usepackage[utf8]{inputenc}
\usepackage[russian,english]{babel}
\usepackage{titling}
\usepackage{amsmath}
\usepackage{float}
\usepackage{graphicx}
\usepackage{enumitem}
\usepackage{amssymb}
\usepackage[super]{nth}
\usepackage{everysel}
\usepackage{ragged2e}
\usepackage{geometry}
\usepackage{multicol}
\usepackage{fancyhdr}
\usepackage{cancel}
\usepackage{siunitx}
\usepackage{physics}
\usepackage{tikz}
\usepackage{mathdots}
\usepackage{yhmath}
\usepackage{cancel}
\usepackage{color}
\usepackage{array}
\usepackage{multirow}
\usepackage{gensymb}
\usepackage{tabularx}
\usepackage{extarrows}
\usepackage{booktabs}
\usepackage{lastpage}
\usetikzlibrary{fadings}
\usetikzlibrary{patterns}
\usetikzlibrary{shadows.blur}
\usetikzlibrary{shapes}

\geometry{top=1.0in,bottom=1.0in,left=1.0in,right=1.0in}
\newcommand{\subtitle}[1]{%
  \posttitle{%
    \par\end{center}
    \begin{center}\large#1\end{center}
    \vskip0.5em}%

}
\usepackage{hyperref}
\hypersetup{
colorlinks=true,
linkcolor=blue,
filecolor=magenta,      
urlcolor=blue,
citecolor=blue,
}


\title{Lecture 2}
\date{\today}
\author{Michael Brodskiy\\ \small Professor: G. Fiete}

\begin{document}

\maketitle

\begin{itemize}

  \item Predictions for experiment: what are the possible results of the measurement of spin projected onto an arbitrary direction, $\hat{n}$, and what are the predicted probabilities?

  \item An operator is a mathematical object that acts on a ket and transforms it into a new ket:

    $$A\ket{\psi}=\ket{\phi}$$

  \item A ket that is not changed by the operator except to be multiplied by a constant is an eigenvector (below, $a$ is an eigenvalue):

    $$A\ket{\psi}=a\ket{\psi}$$

  \item A physical observable is represented mathematically by an operator, $A$, that acts on kets

  \item The only possible result of a measurement of an observable is one of the eigenvalues, $a_n$, of the corresponding operator, $A$. Thus, the equatiion below is an eigenvalue equation:

    $$A\ket{\psi}=a\ket{\psi}$$

    \begin{itemize}

      \item We have observed such an equation already:

        $$S_z\ket{+}=\frac{\hbar}{2}\ket{+}$$
        $$S_z\ket{-}=\frac{-\hbar}{2}\ket{-}$$

      \item Since we had the matrix representations of the up and down states, there must also be one for the operator $S_z$:

        $$\ket{+}=\left( \begin{matrix} 0\\ 1\end{matrix} \right)\quad\text{ and }\quad\ket{-}=\left( \begin{matrix} 0\\ 1\end{matrix} \right)$$
        $$S_z=\left( \begin{matrix} a & b\\ c & d\end{matrix} \right)$$

      \item From this, we get:

        $$S_z\ket{+}\Rightarrow\left( \begin{matrix} a & b\\c & d\end{matrix} \right)\left( \begin{matrix} 1\\ 0\end{matrix} \right)=\left( \begin{matrix}a\\c\end{matrix} \right)=\frac{\hbar}{2}\left( \begin{matrix} 1\\ 0 \end{matrix}\right)$$
        $$S_z\ket{-}\Rightarrow\left( \begin{matrix} a & b\\c & d\end{matrix} \right)\left( \begin{matrix} 0\\ 1\end{matrix} \right)=\left( \begin{matrix}b\\d\end{matrix} \right)=\frac{-\hbar}{2}\left( \begin{matrix} 0\\ 1 \end{matrix}\right)$$

        \item We can thus conclude:

          $$S_z=\frac{\hbar}{2}\left( \begin{matrix} 1 & 0\\ 0 & -1\end{matrix}  \right)$$

        \item We can observe that an operator is always diagonal in its own basis; eigenvectors are unit vectors in their own basis

    \end{itemize}

  \item Matrix Representation of Operators

    \begin{itemize}

      \item A general operator may be expressed as:

        $$A=\left( \begin{matrix} A_{11} & A_{12} & A_{13} & \cdots\\ A_{21} & A_{22} & A_{23} & \cdots\\ A_{31} & A_{32} & A_{33} & \cdots\\ \vdots & \vdots & \vdots & \ddots \end{matrix} \right)$$

      \item This refers to:

        $$A_{ij}=\bra{i | A}\ket{j}$$

      \item The action on a general ket is:

        $$A\ket{\psi}=\left( \begin{matrix} A_{11} & A_{12} & A_{13} & \cdots\\ A_{21} & A_{22} & A_{23} & \cdots\\ A_{31} & A_{32} & A_{33} & \cdots\\ \vdots & \vdots & \vdots & \ddots \end{matrix} \right)\left( \begin{matrix} c_1\\c_2\\c_3\\\vdots\end{matrix} \right)=\left( \begin{matrix} A_{11}c_1+A_{12}c_2+A_{13}c_3+\cdots\\A_{21}c_1+A_{22}c_2+A_{23}c_3+\cdots\\A_{31}c_1+A_{32}c_2+A_{33}c_3+\cdots\\\vdots\end{matrix} \right)$$

      \item If we can write the new ket, we get:

        $$\ket{\phi}=A\ket{\psi}=\sum_i b_i\ket{i}=\sum_i\sum_j A_{ij}c_j$$

    \end{itemize}

  \item Diagonalization of Operators

    \begin{itemize}

      \item Generally, one knows the matrix representation of an operator but wishes to know the possible results of a measurement

      \item The eigenvalue equation:

        $$A\ket{a_n}=a_n\ket{a_n}$$

      \item which in the two-dimensional Hilbert space is:

        $$\left( \begin{matrix} A_{11} & A_{12}\\A_{21} & A_{22}\end{matrix} \right)\left( \begin{matrix} c_1\\c_2\end{matrix} \right)=a_n\left( \begin{matrix} c_1\\c_2\end{matrix} \right)$$

      \item where $c_1,c_2$ are unknown coefficients of $\ket{a_n}$

      \item We can get the following equations:

        $$\begin{matrix} (A_{11}-a_n)c_1+A_{12}c_2=0\\ A_{21}c_1+(A_{22}-a_n)c_2=0\end{matrix}$$

        \begin{itemize}

          \item This has solutions for unknowns $c_1$ and $c_2$ only when the determinant of the coefficients vanishes:

            $$\Big|\begin{matrix} A_{11}-a_n & A_{12}\\ A_{21} & A_{22}-a_n\end{matrix}\Big|=0$$

          \item This can be written in terms of the identity matrix as:

            $$\det(A-\lambda\mathbb{1})=0$$

            \begin{itemize}

              \item Where:

                $$\mathbb{1}=\left( \begin{matrix} 1 & 0\\ 0 & 1\end{matrix} \right)$$

            \end{itemize}

        \end{itemize}

    \end{itemize}

  \item Matrix Representation Summary:

    \begin{itemize}

      \item We can write our observables as:

        $$S_x=\frac{\hbar}{2}\left(\begin{matrix} 0 & 1\\ 1 & 0 \end{matrix}\right)$$
        $$S_y=\frac{\hbar}{2}\left(\begin{matrix}0 & -i\\ i & 0 \end{matrix}\right)$$
        $$S_x=\frac{\hbar}{2}\left(\begin{matrix}1 & 0\\ 0 & -1 \end{matrix}\right)$$

      \item And we can write our kets as:

        $$\ket{\pm}_x=\frac{1}{\sqrt{2}}\left( \begin{matrix}1\\\pm1\end{matrix} \right)$$
        $$\ket{\pm}_y=\frac{1}{\sqrt{2}}\left( \begin{matrix}1\\\pm i\end{matrix} \right)$$

    \end{itemize}

  \item Spin Component in a General Direction

    \begin{itemize}

      \item In Cartesian, the unit vector $\hat{n}$ is:

        $$\hat{n}=\sin(\theta)\cos(\phi)\hat{i}+\sin(\theta)\sin(\phi)\hat{j}+\cos(\theta)\hat{k}$$

      \item The spin components along this direction are found by projection of spin onto this vector:

        $$\vec{S}\cdot\hat{n}=S_x\sin(\theta)\cos(\phi)+S_y\sin(\theta)\sin(\phi)+S_z\cos(\theta)$$

        \begin{itemize}

          \item This is equivalent to:

            $$\frac{\hbar}{2}\left( \begin{matrix} \cos(\theta) & \sin(\theta)e^{-i\phi}\\ \sin(\theta)e^{i\phi} & -\cos(\theta)\end{matrix}\right)$$

        \end{itemize}

      \item Following the diagonalization procedure, the eigenvalues are $\pm\hbar/2$ with eigenvalues:

        $$\ket{+}_n=\cos\left( \frac{\theta}{2} \right)\ket{+}+\sin\left( \frac{\theta}{2} \right)e^{i\phi}\ket{-}$$
        $$\ket{-}_n=\sin\left( \frac{\theta}{2} \right)\ket{+}-\cos\left( \frac{\theta}{2} \right)e^{i\phi}\ket{-}$$

      \item This can represent any possible ket in a spin-1/2 system if one allows for all possible angles:

        $$0\leq\theta <\pi\quad\text{ and }\quad0\leq\phi< 2\pi$$

      \item Using the above, we can find the probability of a cascade of analyzers with the $\hat{n}$ direction and then $x$ direction gives us:

        $$P_{+x}=|\,_x\braket{+}_n|^2=\frac{1}{2}[1+\sin(\theta)\cos(\phi)]$$
        $$P_{-x}=|\,_x\bra{-}\ket{+}_n|^2=\frac{1}{2}[1-\sin(\theta)\cos(\phi)]$$

    \end{itemize}

\end{itemize}

\end{document}

