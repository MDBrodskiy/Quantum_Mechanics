%%%%%%%%%%%%%%%%%%%%%%%%%%%%%%%%%%%%%%%%%%%%%%%%%%%%%%%%%%%%%%%%%%%%%%%%%%%%%%%%%%%%%%%%%%%%%%%%%%%%%%%%%%%%%%%%%%%%%%%%%%%%%%%%%%%%%%%%%%%%%%%%%%%%%%%%%%%%%%%%%%%
% Written By Michael Brodskiy
% Class: Quantum Mechanics
% Professor: G. Fiete
%%%%%%%%%%%%%%%%%%%%%%%%%%%%%%%%%%%%%%%%%%%%%%%%%%%%%%%%%%%%%%%%%%%%%%%%%%%%%%%%%%%%%%%%%%%%%%%%%%%%%%%%%%%%%%%%%%%%%%%%%%%%%%%%%%%%%%%%%%%%%%%%%%%%%%%%%%%%%%%%%%%

\documentclass[12pt]{article} 
\usepackage{alphalph}
\usepackage[utf8]{inputenc}
\usepackage[russian,english]{babel}
\usepackage{titling}
\usepackage{amsmath}
\usepackage{float}
\usepackage{graphicx}
\usepackage{enumitem}
\usepackage{amssymb}
\usepackage[super]{nth}
\usepackage{everysel}
\usepackage{ragged2e}
\usepackage{geometry}
\usepackage{multicol}
\usepackage{fancyhdr}
\usepackage{cancel}
\usepackage{siunitx}
\usepackage{physics}
\usepackage{tikz}
\usepackage{mathdots}
\usepackage{yhmath}
\usepackage{cancel}
\usepackage{color}
\usepackage{array}
\usepackage{multirow}
\usepackage{gensymb}
\usepackage{tabularx}
\usepackage{extarrows}
\usepackage{booktabs}
\usepackage{lastpage}
\usetikzlibrary{fadings}
\usetikzlibrary{patterns}
\usetikzlibrary{shadows.blur}
\usetikzlibrary{shapes}

\geometry{top=1.0in,bottom=1.0in,left=1.0in,right=1.0in}
\newcommand{\subtitle}[1]{%
  \posttitle{%
    \par\end{center}
    \begin{center}\large#1\end{center}
    \vskip0.5em}%

}
\usepackage{hyperref}
\hypersetup{
colorlinks=true,
linkcolor=blue,
filecolor=magenta,      
urlcolor=blue,
citecolor=blue,
}


\title{Lecture 5}
\date{\today}
\author{Michael Brodskiy\\ \small Professor: G. Fiete}

\begin{document}

\maketitle

\begin{itemize}

  \item Associated Legendre Functions

    \begin{itemize}

      \item For $l=0,1,2,3\cdots$

        $$\left[(1-z^2)\frac{d^2}{dz^2}-2z\frac{d}{dz}+l(l+1)-\frac{m^2}{1-z^2}\right]P(z)0$$

      \item With:

        $$P_l^m(z)=P_l^{-m}(z)=(1-z^2)^{m/2}\frac{d^m}{dz^m}P_l(z)$$

      \item Since $P_l(z)$ is an $l$-th order polynomial, then $P_l^m(z)$ vanishes if $m>l$

      \item Properties of $P_l^m(z)$:

        \begin{itemize}

          \item $P_l^m(z)=0$ if $|m|>l$

          \item $P_l^m(z)=P_l^{-m}(z)$

          \item $P_l^m(\pm1)=0$ for $m\neq0$

          \item $P_l^m(-z)=(-1)^{l-m}P_l^{m}(z)$

          \item $\displaystyle \int_{-1}^1 P_l^m(z)P_q^m(z)\,dz=\frac{2}{2l+2}\cdot\frac{(l+m)!}{(l-m)!}\delta_{lq}$

        \end{itemize}

      \item We can obtain our $\Theta(\theta)$ function as:

        $$\Theta_l^m(\theta)=(-1)^m\frac{(2l+1)}{2}\frac{(l-m)!}{(l+m)!}P_l^k(\cos(\theta)),\quad m\geq0$$

        And:

        $$\Theta_l^{-m}(\theta)=(-1)^m\Theta_l^m(\theta)$$

      \item Here, we arrive at a point where the associated Legendre Functions are needed:

        $$P_o^o=1$$
        $$P_1^o=\cos(\theta)$$
        $$P_1^1=\sin(\theta)$$
        $$P_2^o=\frac{1}{2}(3\cos^3(\theta)-1)$$
        $$P_3^o=\frac{1}{2}(5\cos^5(\theta)-3\cos(\theta))$$

      \item Spherical Harmonics

        $$Y_l^m(\theta,\phi)=(-1)^{(m+|m|)/2}\sqrt{\frac{(2l+1)}{4\pi}\frac{(l-|m|)!}{(l+|m|)!}}P_l^m(\cos(\theta))e^{im\phi}$$
        $$Y_l^m(\theta,\phi)=(-1)(Y_l^m(\theta,\phi))^*$$

        Our first few values may be written as:

        $$Y_o^o(\theta,\phi)=\frac{1}{\sqrt{4\pi}}$$
        $$Y_1^o(\theta,\phi)=\sqrt{\frac{3}{4\pi}}\cos(\theta)$$
        $$Y_1^{\pm1}(\theta,\phi)=\mp\sqrt{\frac{3}{8\pi}}\sin(\theta)e^{\pm i\phi}$$

      \item Important Properties:

        \begin{itemize}

          \item Orthonormality:

            $$\bra{l_1m_1}\ket{l_2m_2}=\delta_{l_1l_2}\delta_{m_1m_2}$$

          \item Completeness:

            $$\psi(\theta,\phi)=\sum_{l=0}^{\infty}\sum_{m=-l}^l C_{lm}Y_l^m(\theta,\phi)$$
            $$C_{lm}=\bra{lm}\ket{\psi}=\int_0^{2\pi}\int_0^{\pi}(Y_l^m(\theta,\phi))^*\psi(\theta,\phi)\,d\Omega$$

          \item Parity:

            $$Y_l^m(\pi-\theta,\phi+\pi)=(-1)^lY_l^m(\theta,\phi)$$

        \end{itemize}

    \end{itemize}

  \item The Radial Component

    \begin{itemize}

      \item We may consider a general case:

        $$R_n(\rho)=\rho^lH(\rho)e^{-\gamma\rho}$$

        \begin{itemize}

          \item Where $\rho=r/a$ with $r$ as the radial distance and $a$ as some scaling factor

          \item Furthermore, we have:

            $$\gamma^2=\frac{E}{\left( \frac{\hbar^2}{2\mu a^2} \right)}\Rightarrow E_n=-\frac{1}{2n^2}\left( \frac{Ze^2}{4\pi\varepsilon_o} \right)^2\frac{\mu}{\hbar^2}$$

          \item Our constraints then become:

            $$n=1,2,\cdots,\infty$$
            $$l=0,1,\cdots,n-1$$
            $$m=-l,-l+1,\cdots,l-1,l$$

        \end{itemize}

    \end{itemize}

  \item Hydrogen Energies and Spectrum

    \begin{itemize}

      \item The principal quantum number, $n$, is also called the shell number

      \item We may observe that, as $n\to\infty$ we find the ionization limit

        \begin{itemize}

          \item Note, $E$ does not depend on $m$

        \end{itemize}

      \item For an electron (with a rest energy of $511[\si{\kilo eV}]$), we may find:

        $$E_n=-\frac{13.6}{n^2}$$

      \item The Bohr Radius becomes:

        $$a_o=\frac{4\pi\varepsilon_o\hbar^2}{m_ee^2}$$

      \item We may write the energy scale as:

        $$E_n=-\frac{1}{2n^2}\left( \frac{e^2}{4\pi\varepsilon_oa_o} \right)$$

      \item Noteworthy Features:

        \begin{itemize}

          \item Infinite number of bound states since the Coulomb potential falls of slowly for $r\to\infty$ (finite square well in 3D has only a finite number of bound states)

          \item We compute the degeneracy of $E_n$ by counting up all possible values:

            $$\sum_{l=0}^{n-1}(2l+1)=2\sum_{l=0}^{n-1} l +\sum_{l=0}^{n-1}1$$
            $$\sum_{l=0}^{n-1}(2l+1)= n(n-1)+n$$
            $$\sum_{l=0}^{n-1}(2l+1)= n^2$$

          \item If one includes the spin of the $e^-$ atom, the total degeneracy is $2n^2$

          \item $m$ degeneracy is a result of spherical symmetry, and is removed if an electric field or magnetic field is applied

          \item $l$ degeneracy is a result of the $1/r$ potential and is removed if this changes

          \item The energies of emitted or absorbed light can be obtained as:

            $$E_{photon}=\Delta E_{fi}=|E_f-E_i|$$
            $$|E_f-E_i|=\frac{1}{2}(m_ec^2)\left( \frac{e^2}{4\pi\varepsilon_o\hbar c} \right)^2\Big|\frac{1}{n_i^2}-\frac{1}{n^2_f}\Big|$$

          \item Furthermore, we know:

            $$E_{photon}=\hbar\omega=hv=\frac{hc}{\lambda}$$

          \item We may thus conclude:

            $$\frac{1}{\lambda}=\frac{m_e}{4\pi\hbar^3c}\left( \frac{e^2}{4\pi\varepsilon_o} \right)^2\Big|\frac{1}{n_i^2}-\frac{1}{n^2_f}\Big|$$

          \item Not all transitions are allowed in the hydrogen atom; transitions require a non-zero value of $\langle n_fl_fm_f|V_{int}|n_il_im_i\rangle$

          \item For electromagnetic interactions, the selection rules, which follow from conservation of angular momentum, are:

            $$\Delta l=l_f-l_i=\pm1$$
            $$\Delta m=m_f-m_i=0,\pm1$$

        \end{itemize}

    \end{itemize}

  \item The Radial Wave Functions

    \begin{itemize}

      \item Using:

        $$a=\frac{4\pi\varepsilon_o\hbar^2}{m_eZe^2}=\frac{a_o}{Z}$$
        $$\gamma=\frac{1}{n}$$
        $$\rho=\frac{r}{a}=\frac{Zr}{a_o}\Rightarrow R_{nl}(r)=\left( \frac{Zr}{a_o} \right)^le^{-Zr/na_o}H\left(\frac{Zr}{a_o}\right)$$

    \end{itemize}

  \item The Overall Hydrogen Wave Function

    $$\ket{nlm}=\psi_{nlm}(r,\theta,\phi)=R_{nl}(r)Y_{l}^m(\theta,\phi)$$

    \begin{itemize}

      \item We then establish the following relationships:

        $$H\ket{nlm}=-\frac{13.6}{n^2}\ket{nlm}$$
        $$\vec{L}^2\ket{nlm}=l(l+1)\hbar^2\ket{nlm}$$
        $$L_z\ket{nlm}=m\hbar\ket{nlm}$$

    \end{itemize}

\end{itemize}

\end{document}

