%%%%%%%%%%%%%%%%%%%%%%%%%%%%%%%%%%%%%%%%%%%%%%%%%%%%%%%%%%%%%%%%%%%%%%%%%%%%%%%%%%%%%%%%%%%%%%%%%%%%%%%%%%%%%%%%%%%%%%%%%%%%%%%%%%%%%%%%%%%%%%%%%%%%%%%%%%%%%%%%%%%
% Written By Michael Brodskiy
% Class: Quantum Mechanics
% Professor: G. Fiete
%%%%%%%%%%%%%%%%%%%%%%%%%%%%%%%%%%%%%%%%%%%%%%%%%%%%%%%%%%%%%%%%%%%%%%%%%%%%%%%%%%%%%%%%%%%%%%%%%%%%%%%%%%%%%%%%%%%%%%%%%%%%%%%%%%%%%%%%%%%%%%%%%%%%%%%%%%%%%%%%%%%

\documentclass[12pt]{article} 
\usepackage{alphalph}
\usepackage[utf8]{inputenc}
\usepackage[russian,english]{babel}
\usepackage{titling}
\usepackage{amsmath}
\usepackage{float}
\usepackage{graphicx}
\usepackage{enumitem}
\usepackage{amssymb}
\usepackage[super]{nth}
\usepackage{everysel}
\usepackage{ragged2e}
\usepackage{geometry}
\usepackage{multicol}
\usepackage{fancyhdr}
\usepackage{cancel}
\usepackage{siunitx}
\usepackage{physics}
\usepackage{tikz}
\usepackage{mathdots}
\usepackage{yhmath}
\usepackage{cancel}
\usepackage{color}
\usepackage{array}
\usepackage{multirow}
\usepackage{gensymb}
\usepackage{tabularx}
\usepackage{extarrows}
\usepackage{booktabs}
\usepackage{lastpage}
\usetikzlibrary{fadings}
\usetikzlibrary{patterns}
\usetikzlibrary{shadows.blur}
\usetikzlibrary{shapes}

\geometry{top=1.0in,bottom=1.0in,left=1.0in,right=1.0in}
\newcommand{\subtitle}[1]{%
  \posttitle{%
    \par\end{center}
    \begin{center}\large#1\end{center}
    \vskip0.5em}%

}
\usepackage{hyperref}
\hypersetup{
colorlinks=true,
linkcolor=blue,
filecolor=magenta,      
urlcolor=blue,
citecolor=blue,
}


\title{Lecture 3}
\date{\today}
\author{Michael Brodskiy\\ \small Professor: G. Fiete}

\begin{document}

\maketitle

\begin{itemize}

  \item Hermitian Operators

    \begin{itemize}

      \item So far, we have only considered operators acting on kets:

        $$\ket{\phi}=A\ket{\psi}$$

      \item If the operator acts on a bra it must act to the left:

        $$\bra{\epsilon}=\bra{\psi}A$$

      \item However, the bra, $\bra{\epsilon}$, is \underline{not} the bra that corresponds to the ket, $\ket{\phi}=A\ket{\psi}$

      \item The bra $\bra{\phi}$ is found by defining a new operator $A^+$ that obeys:

        $$\bra{\phi}=\bra{\psi}A^{+}$$

        \begin{itemize}

          \item $A^{+}$ is called the Hermitian adjoint of $A$. Consider the inner product:

            $$\bra{\phi}\ket{\beta}=\bra{\beta}\ket{\phi}^*$$
            $$\bra{\psi|A^{+}}\ket{\beta}=(\bra{\beta|A}\ket{\psi})^*$$

          \item This relates the matrix elements of $A$ and $A^{+}$
            
          \item Therefore, $A^{+}$ is found by transposing and complex conjugating the matrix representing $A$

        \end{itemize}

      \item An operator, $A$, is Hermitian if it is equal to its Hermitian adjoint, $A^{+}$

      \item If an operator is Hermitian, then its bra, $\bra{\psi}A$ is equal to the bra $\bra{\phi}$ that corresponds to the ket $\ket{\phi}=A\ket{\psi}$

        \begin{itemize}

          \item In quantum mechanics, all operators that correspond to physical observables are Hermitian

        \end{itemize}

      \item Hermitian matrices have real eigenvalues, which ensures results of measurements are always real-values

      \item The eigenvectors of Hermitian matrices comprise a complete set of basis states, which ensures the eigenvectors of any observable are a valid basis

    \end{itemize}

  \item Projection Operators

    \begin{itemize}

      \item Recall for a spin-1/2 system we had the identity relation:

        $$\ket{+}\bra{+}+\ket{-}\bra{-}=\mathbb{1}$$

      \item We can express this in matrix notation as:

        $$\left( \begin{matrix}1\\0\end{matrix} \right)(1\quad 0)+\left( \begin{matrix}0\\1\end{matrix} \right)(0\quad 1)=\left( \begin{matrix}1 & 0\\0 & 1\end{matrix} \right)$$

      \item This gives us the 2x2 identity matrix

      \item The individual operators, $\ket{+}\bra{+}$ and $\ket{-}\bra{-}$, are called projection operators:

        $$P_+=\ket{+}\bra{+}=\left( \begin{matrix}1&0\\0&0\end{matrix} \right)$$
        $$P_-=\ket{-}\bra{-}=\left( \begin{matrix}0&0\\0&1\end{matrix} \right)$$

      \item Thus, for a general state, we may write $P_++P_-=\mathbb{1}$

      \item From here, we may write:

        $$P_+\ket{\psi}=\ket{+}\bra{+}\ket{\psi}=(\bra{+}\ket{\psi})\ket{+}$$
        $$P_-\ket{\psi}=\ket{-}\bra{-}\ket{\psi}=(\bra{-}\ket{\psi})\ket{-}$$

      \item The effect of the projection operator on a given state is to produce a new, normalized state

        $$\ket{\psi'}=P_+\ket{\psi}$$

      \item The projection postulate thus becomes:

        $$\ket{\psi'}=\frac{P_+\ket{\psi}}{\sqrt{\bra{\psi|P_+}\ket{\psi}}}=\ket{+}$$
        
      \item This indicates a ``collapse'' of the quantum state vector

    \end{itemize}

  \item Measurement

    \begin{itemize}

      \item In quantum mechanics, one must perform multiple identical measurements on identically prepared systems to infer the probabilities of outcomes

      \item For example, if one performs $N$ measurements of the projections of $\ket{\psi}$ and obtains $+\hbar/2$ $N_+$ times, then:

        $$\lim_{N\to\infty}\frac{N_+}{N}=|\bra{+}\ket{\psi}|^2$$

      \item It is useful to characterize statistical data sets by their mean and standard deviation

        $$<S_z>=\frac{\hbar}{2}P_++\left(-\frac{\hbar}{2}\right)P_-=\bra{\psi|S_z}\ket{\psi}=\sum_{n}a_nP_{a_n}$$

        \begin{itemize}

          \item We may observe that this is the sum of the eigenvalues multiplied by the probability of getting said eigenvalue

          \item For the spin-1/2 system with $\ket{+}$ we get:

            $$<S_z>=\bra{+|S_z}\ket{+}=\bra{+|\hbar/2}\ket{+}=\frac{\hbar}{2}\braket{+}=\frac{\hbar}{2}$$

          \item Similarly, we may apply $\ket{+}_x$ to observe:

            $$<S_z>=\,_x\bra{+|S_z}\ket{+}_x=\,_x\bra{+|\hbar/2}\ket{+}_x=\frac{\hbar}{4}(1-1)=0$$

        \end{itemize}

      \item It is common to characterize the standard deviation by the root-mean-square:

        $$\Delta A=\sqrt{<(A-<A>)^2>}=\sqrt{<A^2>-<A>^2}$$

        \begin{itemize}

          \item From the above, it is important to note that in the general case:

            $$<A^2>\neq <A>^2$$

        \end{itemize}

      \item The square of an operator acts twice on the state:

        $$A^2\ket{\psi}=A(A\ket{\psi})$$

        \begin{itemize}

          \item Similarly, if we consider $S_z$, we may write:

            $$<S_z^2>=\bra{+|S_z^2}\ket{+}=\bra{+|(\hbar/2)^2}\ket{+}=\frac{\hbar^2}{4}$$

          \item Thus, we may write:

            $$\Delta S_z=0$$

          \item Now, we check a different orientation:

            $$_x\bra{+|S_z^2}\ket{+}_x=\frac{\hbar^2}{4}$$

          \item Thus, we may see:

            $$\Delta_xS_z=\frac{h}{2}$$

        \end{itemize}

    \end{itemize}

  \item Commuting Observables

    \begin{itemize}

      \item Two incompatible observables may be identified with a commutator:

        $$[A,B]=AB-BA$$

      \item If $[A,B]=0$, operators (observables) are said to commute and are compatible

      \item Assuming $[A,B]=0$, then:

        $$AB=BA$$

        \begin{itemize}

          \item Let $\ket{a}$ be an eigenstate of $A$ with eigenvalue $a$:

            $$A\ket{a}=a\ket{a}$$

          \item Then we can say:

            $$BA\ket{a}=aB\ket{a}$$

          \item This can be expanded:

            $$AB\ket{a}=A(B\ket{a})=a\ket{a'}$$

            \begin{itemize}

              \item Here, we make $B\ket{a}$ an eigenstate of $A$ with eigenvalue $a$

            \end{itemize}

          \item Assuming each eigenvalue has a unique eigenstate, then $B\ket{a}$ must be a scalar multiple of $\ket{a}$

            $$B\ket{a}=b\ket{a}$$

          \item Thus, we conclude:

            $$\text{If }[A,B]=0,\text{ }A\text{ and }B\text{ have simultaneous sets of eigenstates}$$

          \item Conversely, if two operators do not commute, they are incompatible and can not be known simultaneously, like $S_z$ and $S_x$

        \end{itemize}

    \end{itemize}

  \item Uncertainty Principle

    \begin{itemize}

      \item There is an intimate connection between the commutator of two observables and the possible precision of measurements of each:

        $$\Delta A\Delta B\geq \frac{1}{2}|<[A,B]>|$$

      \item Applying to Stern-Gerlach, we may write:

        $$\Delta S_x\Delta S_y\geq \frac{1}{2}|<[S_x,S_y]>|$$
        $$\Delta S_x\Delta S_y\geq \frac{\hbar}{2}|<S_z>|$$

        \begin{itemize}

          \item Applying this to $\ket{+}$, we find:

            $$\Delta S_x\Delta S_y\geq \left( \frac{\hbar}{2} \right)^2$$

            \begin{itemize}

              \item This implies that the individual components are both non-zero

              \item Therefore, we can not know spin components of either component absolutely

              \item As a result, one can not say the spin points in a given direction

            \end{itemize}

        \end{itemize}

    \end{itemize}

  \item The $\vec{S}$ Operation

    \begin{itemize}

      \item Let us begin by writing:

        $$\vec{S}^2=S_x^2+S_y^2+S_z^2$$

      \item It points in no direction in space. We can calculate using matrix notation:

        $$\vec{S}^2=\left( \frac{\hbar}{2} \right)^2\left[\left( \begin{matrix} 0 & 1\\1 & 0\end{matrix} \right)\left( \begin{matrix} 0 & 1\\1 & 0\end{matrix} \right)+\left( \begin{matrix} 0 & -i\\i & 0\end{matrix} \right)\left( \begin{matrix} 0 & -i\\i & 0\end{matrix} \right)+\left( \begin{matrix} 1 & 0\\0 & -1\end{matrix} \right)\left( \begin{matrix} 1 & 0\\0 & -1\end{matrix} \right)\right]$$
        $$\vec{S}^2=\left( \frac{\hbar}{2} \right)^2\left[ \mathbb{1}+\mathbb{1}+\mathbb{1} \right]$$
        $$\vec{S}^2=\frac{3\hbar^2}{4}\mathbb{1}$$

      \item For any state $\ket{\psi}$ in the Hilbert space $S=1/2$:

        $$\vec{S}^2\ket{\psi}=\frac{3}{4}\hbar^2\ket{\psi}$$

      \item So, we may conclude:

        $$<\vec{S}^2>=\frac{3}{4}\hbar^2$$

      \item This would imply the ``length'' of the spin vector is:

        $$|\vec{S}|=\sqrt{<\vec{S}^2>}=\sqrt{3}(\hbar/2)$$

      \item Thus, this value is greater than the measured component, $\hbar/2$, implying that the spin vector is never fully aligned with any axis

    \end{itemize}

  \item Spin-1 Systems

    \begin{itemize}

      \item The Stern-Gerlach experiment produces 3 beams corresponding to $z$-axis projections $+\hbar$, $0$, and $-\hbar$:

        $$\ket{1},\ket{0},\ket{-1}\Longrightarrow \left\{\begin{array}{ll} S_z\ket{1}&=\hbar\ket{1}\\S_z\ket{0}&=0\hbar\ket{0}\\S_z\ket{-1}&= -\hbar\ket{-1}\end{array}$$

        \item Recall eigenvectors are unit vectors in their own basis and an operator is always diagonal in its own basis. This gives us:

          $$\ket{1}=\left( \begin{matrix}1\\0\\0\end{matrix} \right)\quad\ket{0}=\left( \begin{matrix}0\\1\\0\end{matrix} \right)\quad\ket{-1}=\left( \begin{matrix}0\\0\\1\end{matrix} \right)$$

          \begin{itemize}

            \item This then gives us:

              $$S_x=\hbar\left(\begin{matrix} 1 & 0 & 0\\0 & 0 & 0\\ 0 & 0 & -1\end{matrix}  \right)$$

          \end{itemize}

        \item We can write the $x$ orientation as:

          $$\ket{1}_x=\frac{1}{2}\ket{1}+\frac{1}{\sqrt{2}}\ket{0}+\frac{1}{2}\ket{-1}$$
          $$\ket{0}_x=\frac{1}{\sqrt{2}}\ket{1}-\frac{1}{\sqrt{2}}\ket{-1}$$
          $$\ket{-1}_x=\frac{1}{2}\ket{1}-\frac{1}{\sqrt{2}}\ket{0}+\frac{1}{2}\ket{-1}$$

        \item We then do the same for the $y$ orientation:

          $$\ket{1}_y=\frac{1}{2}\ket{1}+\frac{i}{\sqrt{2}}\ket{0}-\frac{1}{2}\ket{-1}$$
          $$\ket{0}_x=\frac{1}{\sqrt{2}}\ket{1}+\frac{1}{\sqrt{2}}\ket{-1}$$
          $$\ket{-1}_x=\frac{1}{2}\ket{1}-\frac{i}{\sqrt{2}}\ket{0}-\frac{1}{2}\ket{-1}$$

        \item The operators can be written as:

          $$S_x=\frac{\hbar}{\sqrt{2}}\left( \begin{matrix} 0 & 1 & 0\\ 1 & 0 & 1\\ 0 & 1 & 0\end{matrix} \right)$$
          $$S_y=\frac{\hbar}{\sqrt{2}}\left( \begin{matrix} 0 & -i & 0\\ i & 0 & -i\\ 0 & i & 0\end{matrix} \right)$$

    \end{itemize}

\end{itemize}

\end{document}

