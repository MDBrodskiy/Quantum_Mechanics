%%%%%%%%%%%%%%%%%%%%%%%%%%%%%%%%%%%%%%%%%%%%%%%%%%%%%%%%%%%%%%%%%%%%%%%%%%%%%%%%%%%%%%%%%%%%%%%%%%%%%%%%%%%%%%%%%%%%%%%%%%%%%%%%%%%%%%%%%%%%%%%%%%%%%%%%%%%%%%%%%%%
% Written By Michael Brodskiy
% Class: Quantum Mechanics
% Professor: G. Fiete
%%%%%%%%%%%%%%%%%%%%%%%%%%%%%%%%%%%%%%%%%%%%%%%%%%%%%%%%%%%%%%%%%%%%%%%%%%%%%%%%%%%%%%%%%%%%%%%%%%%%%%%%%%%%%%%%%%%%%%%%%%%%%%%%%%%%%%%%%%%%%%%%%%%%%%%%%%%%%%%%%%%

\include{Includes.tex}

\title{Lecture 3}
\date{\today}
\author{Michael Brodskiy\\ \small Professor: G. Fiete}

\begin{document}

\maketitle

\begin{itemize}

  \item Hermitian Operators

    \begin{itemize}

      \item So far, we have only considered operators acting on kets:

        $$\ket{\phi}=A\ket{\psi}$$

      \item If the operator acts on a bra it must act to the left:

        $$\bra{\epsilon}=\bra{\psi}A$$

      \item However, the bra, $\bra{\epsilon}$, is \underline{not} the bra that corresponds to the ket, $\ket{\phi}=A\ket{\psi}$

      \item The bra $\bra{\phi}$ is found by defining a new operator $A^+$ that obeys:

        $$\bra{\phi}=\bra{\psi}A^{+}$$

        \begin{itemize}

          \item $A^{+}$ is called the Hermitian adjoint of $A$. Consider the inner product:

            $$\bra{\phi}\ket{\beta}=\bra{\beta}\ket{\phi}^*$$
            $$\bra{\psi|A^{+}}\ket{\beta}=(\bra{\beta|A}\ket{\psi})^*$$

          \item This relates the matrix elements of $A$ and $A^{+}$
            
          \item Therefore, $A^{+}$ is found by transposing and complex conjugating the matrix representing $A$

        \end{itemize}

      \item An operator, $A$, is Hermitian if it is equal to its Hermitian adjoint, $A^{+}$

      \item If an operator is Hermitian, then its bra, $\bra{\psi}A$ is equal to the bra $\bra{\phi}$ that corresponds to the ket $\ket{\phi}=A\ket{\psi}$

        \begin{itemize}

          \item In quantum mechanics, all operators that correspond to physical observables are Hermitian

        \end{itemize}

      \item Hermitian matrices have real eigenvalues, which ensures results of measurements are always real-values

      \item The eigenvectors of Hermitian matrices comprise a complete set of basis states, which ensures the eigenvectors of any observable are a valid basis

    \end{itemize}

  \item Projection Operators

    \begin{itemize}

      \item Recall for a spin-1/2 system we had the identity relation:

        $$\ket{+}\bra{+}+\ket{-}\bra{-}=\mathbb{1}$$

      \item We can express this in matrix notation as:

        $$\left( \begin{matrix}1\\0\end{matrix} \right)(1\quad 0)+\left( \begin{matrix}0\\1\end{matrix} \right)(0\quad 1)=\left( \begin{matrix}1 & 0\\0 & 1\end{matrix} \right)$$

      \item This gives us the 2x2 identity matrix

      \item The individual operators, $\ket{+}\bra{+}$ and $\ket{-}\bra{-}$, are called projection operators:

        $$P_+=\ket{+}\bra{+}=\left( \begin{matrix}1&0\\0&0\end{matrix} \right)$$
        $$P_-=\ket{-}\bra{-}=\left( \begin{matrix}0&0\\0&1\end{matrix} \right)$$

      \item Thus, for a general state, we may write $P_++P_-=\mathbb{1}$

      \item From here, we may write:

        $$P_+\ket{\psi}=\ket{+}\bra{+}\ket{\psi}=(\bra{+}\ket{\psi})\ket{+}$$
        $$P_-\ket{\psi}=\ket{-}\bra{-}\ket{\psi}=(\bra{-}\ket{\psi})\ket{-}$$

      \item The effect of the projection operator on a given state is to produce a new, normalized state

        $$\ket{\psi'}=P_+\ket{\psi}$$

      \item The projection postulate thus becomes:

        $$\ket{\psi'}=\frac{P_+\ket{\psi}}{\sqrt{\bra{\psi|P_+}\ket{\psi}}}=\ket{+}$$
        
      \item This indicates a ``collapse'' of the quantum state vector

    \end{itemize}

  \item Measurement

    \begin{itemize}

      \item In quantum mechanics, one must perform multiple identical measurements on identically prepared systems to infer the probabilities of outcomes

      \item For example, if one performs $N$ measurements of the projections of $\ket{\psi}$ and obtains $+\hbar/2$ $N_+$ times, then:

        $$\lim_{N\to\infty}\frac{N_+}{N}=|\bra{+}\ket{\psi}|^2$$

      \item It is useful to characterize statistical data sets by their mean and standard deviation

        $$<S_z>=\frac{\hbar}{2}P_++\left(-\frac{\hbar}{2}\right)P_-=\bra{\psi|S_z}\ket{\psi}$$

    \end{itemize}

\end{itemize}

\end{document}

