%%%%%%%%%%%%%%%%%%%%%%%%%%%%%%%%%%%%%%%%%%%%%%%%%%%%%%%%%%%%%%%%%%%%%%%%%%%%%%%%%%%%%%%%%%%%%%%%%%%%%%%%%%%%%%%%%%%%%%%%%%%%%%%%%%%%%%%%%%%%%%%%%%%%%%%%%%%%%%%%%%%
% Written By Michael Brodskiy
% Class: Quantum Mechanics
% Professor: G. Fiete
%%%%%%%%%%%%%%%%%%%%%%%%%%%%%%%%%%%%%%%%%%%%%%%%%%%%%%%%%%%%%%%%%%%%%%%%%%%%%%%%%%%%%%%%%%%%%%%%%%%%%%%%%%%%%%%%%%%%%%%%%%%%%%%%%%%%%%%%%%%%%%%%%%%%%%%%%%%%%%%%%%%

\documentclass[12pt]{article} 
\usepackage{alphalph}
\usepackage[utf8]{inputenc}
\usepackage[russian,english]{babel}
\usepackage{titling}
\usepackage{amsmath}
\usepackage{float}
\usepackage{graphicx}
\usepackage{enumitem}
\usepackage{amssymb}
\usepackage[super]{nth}
\usepackage{everysel}
\usepackage{ragged2e}
\usepackage{geometry}
\usepackage{multicol}
\usepackage{fancyhdr}
\usepackage{cancel}
\usepackage{siunitx}
\usepackage{physics}
\usepackage{tikz}
\usepackage{mathdots}
\usepackage{yhmath}
\usepackage{cancel}
\usepackage{color}
\usepackage{array}
\usepackage{multirow}
\usepackage{gensymb}
\usepackage{tabularx}
\usepackage{extarrows}
\usepackage{booktabs}
\usepackage{lastpage}
\usetikzlibrary{fadings}
\usetikzlibrary{patterns}
\usetikzlibrary{shadows.blur}
\usetikzlibrary{shapes}

\geometry{top=1.0in,bottom=1.0in,left=1.0in,right=1.0in}
\newcommand{\subtitle}[1]{%
  \posttitle{%
    \par\end{center}
    \begin{center}\large#1\end{center}
    \vskip0.5em}%

}
\usepackage{hyperref}
\hypersetup{
colorlinks=true,
linkcolor=blue,
filecolor=magenta,      
urlcolor=blue,
citecolor=blue,
}


\title{Lecture 3}
\date{\today}
\author{Michael Brodskiy\\ \small Professor: G. Fiete}

\begin{document}

\maketitle

\begin{itemize}

  \item Hermitian Operators

    \begin{itemize}

      \item So far, we have only considered operators acting on kets:

        $$\ket{\phi}=A\ket{\psi}$$

      \item If the operator acts on a bra it must act to the left:

        $$\bra{\epsilon}=\bra{\psi}A$$

      \item However, the bra, $\bra{\epsilon}$, is \underline{not} the bra that corresponds to the ket, $\ket{\phi}=A\ket{\psi}$

      \item The bra $\bra{\phi}$ is found by defining a new operator $A^+$ that obeys:

        $$\bra{\phi}=\bra{\psi}A^{+}$$

        \begin{itemize}

          \item $A^{+}$ is called the Hermitian adjoint of $A$. Consider the inner product:

            $$\bra{\phi}\ket{\beta}=\bra{\beta}\ket{\phi}^*$$
            $$\bra{\psi|A^{+}}\ket{\beta}=(\bra{\beta|A}\ket{\psi})^*$$

          \item This relates the matrix elements of $A$ and $A^{+}$
            
          \item Therefore, $A^{+}$ is found by transposing and complex conjugating the matrix representing $A$

        \end{itemize}

      \item An operator, $A$, is Hermitian if it is equal to its Hermitian adjoint, $A^{+}$

      \item If an operator is Hermitian, then its bra, $\bra{\psi}A$ is equal to the bra $\bra{\phi}$ that corresponds to the ket $\ket{\phi}=A\ket{\psi}$

        \begin{itemize}

          \item In quantum mechanics, all operators that correspond to physical observables are Hermitian

        \end{itemize}

      \item Hermitian matrices have real eigenvalues, which ensures results of measurements are always real-values

      \item The eigenvectors of Hermitian matrices comprise a complete set of basis states, which ensures the eigenvectors of any observable are a valid basis

    \end{itemize}

  \item Projection Operators

    \begin{itemize}

      \item Recall for a spin-1/2 system we had the identity relation:

        $$\ket{+}\bra{+}+\ket{-}\bra{-}=\mathbb{1}$$

      \item We can express this in matrix notation as:

        $$\left( \begin{matrix}1\\0\end{matrix} \right)(1\quad 0)+\left( \begin{matrix}0\\1\end{matrix} \right)(0\quad 1)=\left( \begin{matrix}1 & 0\\0 & 1\end{matrix} \right)$$

      \item This gives us the 2x2 identity matrix

      \item The individual operators, $\ket{+}\bra{+}$ and $\ket{-}\bra{-}$, are called projection operators:

        $$P_+=\ket{+}\bra{+}=\left( \begin{matrix}1&0\\0&0\end{matrix} \right)$$
        $$P_-=\ket{-}\bra{-}=\left( \begin{matrix}0&0\\0&1\end{matrix} \right)$$

      \item Thus, for a general state, we may write $P_++P_-=\mathbb{1}$

      \item From here, we may write:

        $$P_+\ket{\psi}=\ket{+}\bra{+}\ket{\psi}=(\bra{+}\ket{\psi})\ket{+}$$
        $$P_-\ket{\psi}=\ket{-}\bra{-}\ket{\psi}=(\bra{-}\ket{\psi})\ket{-}$$

      \item The effect of the projection operator on a given state is to produce a new, normalized state

        $$\ket{\psi'}=P_+\ket{\psi}$$

      \item The projection postulate thus becomes:

        $$\ket{\psi'}=\frac{P_+\ket{\psi}}{\sqrt{\bra{\psi|P_+}\ket{\psi}}}=\ket{+}$$
        
      \item This indicates a ``collapse'' of the quantum state vector

    \end{itemize}

  \item Measurement

    \begin{itemize}

      \item In quantum mechanics, one must perform multiple identical measurements on identically prepared systems to infer the probabilities of outcomes

      \item For example, if one performs $N$ measurements of the projections of $\ket{\psi}$ and obtains $+\hbar/2$ $N_+$ times, then:

        $$\lim_{N\to\infty}\frac{N_+}{N}=|\bra{+}\ket{\psi}|^2$$

      \item It is useful to characterize statistical data sets by their mean and standard deviation

        $$<S_z>=\frac{\hbar}{2}P_++\left(-\frac{\hbar}{2}\right)P_-=\bra{\psi|S_z}\ket{\psi}=\sum_{n}a_nP_{a_n}$$

        \begin{itemize}

          \item We may observe that this is the sum of the eigenvalues multiplied by the probability of getting said eigenvalue

          \item For the spin-1/2 system with $\ket{+}$ we get:

            $$<S_z>=\bra{+|S_z}\ket{+}=\bra{+|\hbar/2}\ket{+}=\frac{\hbar}{2}\braket{+}=\frac{\hbar}{2}$$

          \item Similarly, we may apply $\ket{+}_x$ to observe:

            $$<S_z>=\,_x\bra{+|S_z}\ket{+}_x=\,_x\bra{+|\hbar/2}\ket{+}_x=\frac{\hbar}{4}(1-1)=0$$

        \end{itemize}

      \item It is common to characterize the standard deviation by the root-mean-square:

        $$\Delta A=\sqrt{<(A-<A>)^2>}=\sqrt{<A^2>-<A>^2}$$

        \begin{itemize}

          \item From the above, it is important to note that in the general case:

            $$<A^2>\neq <A>^2$$

        \end{itemize}

      \item The square of an operator acts twice on the state:

        $$A^2\ket{\psi}=A(A\ket{\psi})$$

        \begin{itemize}

          \item Similarly, if we consider $S_z$, we may write:

            $$<S_z^2>=\bra{+|S_z^2}\ket{+}=\bra{+|(\hbar/2)^2}\ket{+}=\frac{\hbar^2}{4}$$

          \item Thus, we may write:

            $$\Delta S_z=0$$

          \item Now, we check a different orientation:

            $$_x\bra{+|S_z^2}\ket{+}_x=\frac{\hbar^2}{4}$$

          \item Thus, we may see:

            $$\Delta_xS_z=\frac{h}{2}$$

        \end{itemize}

    \end{itemize}

  \item Commuting Observables

    \begin{itemize}

      \item Two incompatible observables may be identified with a commutator:

        $$[A,B]=AB-BA$$

      \item If $[A,B]=0$, operators (observables) are said to commute and are compatible

      \item Assuming $[A,B]=0$, then:

        $$AB=BA$$

        \begin{itemize}

          \item Let $\ket{a}$ be an eigenstate of $A$ with eigenvalue $a$:

            $$A\ket{a}=a\ket{a}$$

          \item Then we can say:

            $$BA\ket{a}=aB\ket{a}$$

          \item This can be expanded:

            $$AB\ket{a}=A(B\ket{a})=a\ket{a'}$$

            \begin{itemize}

              \item Here, we make $B\ket{a}$ an eigenstate of $A$ with eigenvalue $a$

            \end{itemize}

          \item Assuming each eigenvalue has a unique eigenstate, then $B\ket{a}$ must be a scalar multiple of $\ket{a}$

            $$B\ket{a}=b\ket{a}$$

          \item Thus, we conclude:

            $$\text{If }[A,B]=0,\text{ }A\text{ and }B\text{ have simultaneous sets of eigenstates}$$

          \item Conversely, if two operators do not commute, they are incompatible and can not be known simultaneously, like $S_z$ and $S_x$

        \end{itemize}

    \end{itemize}

  \item Uncertainty Principle

    \begin{itemize}

      \item There is an intimate connection between the commutator of two observables and the possible precision of measurements of each:

        $$\Delta A\Delta B\geq \frac{1}{2}|<[A,B]>|$$

      \item Applying to Stern-Gerlach, we may write:

        $$\Delta S_x\Delta S_y\geq \frac{1}{2}|<[S_x,S_y]>|$$
        $$\Delta S_x\Delta S_y\geq \frac{\hbar}{2}|<S_z>|$$

        \begin{itemize}

          \item Applying this to $\ket{+}$, we find:

            $$\Delta S_x\Delta S_y\geq \left( \frac{\hbar}{2} \right)^2$$

            \begin{itemize}

              \item This implies that the individual components are both non-zero

              \item Therefore, we can not know spin components of either component absolutely

              \item As a result, one can not say the spin points in a given direction

            \end{itemize}

        \end{itemize}

    \end{itemize}

  \item The $\vec{S}$ Operation

    \begin{itemize}

      \item Let us begin by writing:

        $$\vec{S}^2=S_x^2+S_y^2+S_z^2$$

      \item It points in no direction in space. We can calculate using matrix notation:

        $$\vec{S}^2=\left( \frac{\hbar}{2} \right)^2\left[\left( \begin{matrix} 0 & 1\\1 & 0\end{matrix} \right)\left( \begin{matrix} 0 & 1\\1 & 0\end{matrix} \right)+\left( \begin{matrix} 0 & -i\\i & 0\end{matrix} \right)\left( \begin{matrix} 0 & -i\\i & 0\end{matrix} \right)+\left( \begin{matrix} 1 & 0\\0 & -1\end{matrix} \right)\left( \begin{matrix} 1 & 0\\0 & -1\end{matrix} \right)\right]$$
        $$\vec{S}^2=\left( \frac{\hbar}{2} \right)^2\left[ \mathbb{1}+\mathbb{1}+\mathbb{1} \right]$$
        $$\vec{S}^2=\frac{3\hbar^2}{4}\mathbb{1}$$

      \item For any state $\ket{\psi}$ in the Hilbert space $S=1/2$:

        $$\vec{S}^2\ket{\psi}=\frac{3}{4}\hbar^2\ket{\psi}$$

      \item So, we may conclude:

        $$<\vec{S}^2>=\frac{3}{4}\hbar^2$$

      \item This would imply the ``length'' of the spin vector is:

        $$|\vec{S}|=\sqrt{<\vec{S}^2>}=\sqrt{3}(\hbar/2)$$

      \item Thus, this value is greater than the measured component, $\hbar/2$, implying that the spin vector is never fully aligned with any axis

    \end{itemize}

  \item Spin-1 Systems

    \begin{itemize}

      \item The Stern-Gerlach experiment produces 3 beams corresponding to $z$-axis projections $+\hbar$, $0$, and $-\hbar$:

        $$\ket{1},\ket{0},\ket{-1}\Longrightarrow \left\{\begin{array}{ll} S_z\ket{1}&=\hbar\ket{1}\\S_z\ket{0}&=0\hbar\ket{0}\\S_z\ket{-1}&= -\hbar\ket{-1}\end{array}$$

        \item Recall eigenvectors are unit vectors in their own basis and an operator is always diagonal in its own basis. This gives us:

          $$\ket{1}=\left( \begin{matrix}1\\0\\0\end{matrix} \right)\quad\ket{0}=\left( \begin{matrix}0\\1\\0\end{matrix} \right)\quad\ket{-1}=\left( \begin{matrix}0\\0\\1\end{matrix} \right)$$

          \begin{itemize}

            \item This then gives us:

              $$S_x=\hbar\left(\begin{matrix} 1 & 0 & 0\\0 & 0 & 0\\ 0 & 0 & -1\end{matrix}  \right)$$

          \end{itemize}

        \item We can write the $x$ orientation as:

          $$\ket{1}_x=\frac{1}{2}\ket{1}+\frac{1}{\sqrt{2}}\ket{0}+\frac{1}{2}\ket{-1}$$
          $$\ket{0}_x=\frac{1}{\sqrt{2}}\ket{1}-\frac{1}{\sqrt{2}}\ket{-1}$$
          $$\ket{-1}_x=\frac{1}{2}\ket{1}-\frac{1}{\sqrt{2}}\ket{0}+\frac{1}{2}\ket{-1}$$

        \item We then do the same for the $y$ orientation:

          $$\ket{1}_y=\frac{1}{2}\ket{1}+\frac{i}{\sqrt{2}}\ket{0}-\frac{1}{2}\ket{-1}$$
          $$\ket{0}_x=\frac{1}{\sqrt{2}}\ket{1}+\frac{1}{\sqrt{2}}\ket{-1}$$
          $$\ket{-1}_x=\frac{1}{2}\ket{1}-\frac{i}{\sqrt{2}}\ket{0}-\frac{1}{2}\ket{-1}$$

        \item The operators can be written as:

          $$S_x=\frac{\hbar}{\sqrt{2}}\left( \begin{matrix} 0 & 1 & 0\\ 1 & 0 & 1\\ 0 & 1 & 0\end{matrix} \right)$$
          $$S_y=\frac{\hbar}{\sqrt{2}}\left( \begin{matrix} 0 & -i & 0\\ i & 0 & -i\\ 0 & i & 0\end{matrix} \right)$$

    \end{itemize}

  \item New Labels

    \begin{itemize}

      \item The allowed values of spin are $S=1/2,1,3/2,2,\cdots$

      \item The number of beams exiting the Stern-Gerlach with such spin is $2S+1$ with a minimum value of $-\hbar S$ and a maximum value  $+\hbar S$

      \item We label the state $\ket{S_m}$

        $$\vec{S}\ket{S_m}=S(S+1)\hbar^2\ket{S_m}$$
        $$S_z\ket{S_m}=m\hbar\ket{S_m}$$
        $$[\vec{S}^2,S_z]=0$$

        \begin{itemize}

          \item Note, $S_m$ is the magnetic quantum number

        \end{itemize}

      \item Schr\"odinger Equation

        \begin{itemize}

          \item The time evolution of a quantum system, is determined by the Hamiltonian of the total energy operator $H(t)$ through the Schr\"odinger equation:

            $$i\hbar\frac{d}{dt}\ket{\psi(t)}=H(t)\ket{\psi(t)}$$

        \end{itemize}

      \item The Energy Eigenvalue Equation

        $$H\ket{E_n}=E_n\ket{E_n}$$

        \begin{itemize}

          \item Where $E_n$ represents the only allowed energies of the system

          \item Since $H$ represents the energy, it is observable, Hermitian, and therefore, its eigenvalues form a complete basis

          \item Since $H$ is the only operator appearing in the Schr\"odinger equation, it is convenient to expand the state vectors in terms of the energy eigenstates:

            $$\ket{\psi}=\sum_{n}\bra{n(t)}\ket{E_n}$$

          \item Substituting the expansion into the Schr\"odinger equation gets us:

            $$i\hbar\frac{d}{dt}\sum_n \bra{n(t)}\ket{E_n}=\sum_n\bra{n(t)E_n}\ket{E_n}$$
            $$\frac{d C_k(t)}{dt}=- i\frac{E_k}{\hbar}C_k(t)$$

          \item We see that a possible solution is:

            $$C_k(t)=C_o(t)e^{-iE_k/\hbar}$$

            \begin{itemize}

              \item Each energy has the same form of the time dependence but a different exponent. At time $t=0$:

                $$\ket{\psi(0)}=\sum_n\bra{n}\ket{E_n}$$

              \item And the time evolution of a time-independent Hamiltonian is:

                $$\ket{\psi(t)}=\sum_n C_ne^{-iE_nt/\hbar}\ket{E_n}$$

                $$H\ket{E_n}=E_n\ket{E_n}$$

              \item Note that the factor $e^{-iE_nt/\hbar}$ has the form $e^{-i\omega t}$ found in many areas of physics, such that:

                $$\omega_n=\frac{E_n}{\hbar}\Rightarrow E_n=\hbar \omega_n$$

              \item Energy eigenstates are called stationary states

              \item If the system is in an energy eigenstate, it remains in that state

            \end{itemize}

          \item The Bohr Frequency

            \begin{itemize}

              \item The Bohr frequency may be found as:

                $$\omega_{21}=\frac{E_2-E_1}{\hbar}$$

            \end{itemize}

        \end{itemize}

    \end{itemize}

  \item Spin Precession

    \begin{itemize}

      \item Time-evolution of $S=1/2$

        $$\vec{\mu}=g\frac{q}{2m}\vec{S}$$

        \begin{itemize}

          \item For an electron:

            $$\vec{\mu}=\frac{e}{m_e}\vec{S}$$

        \end{itemize}

      \item The Hamiltonian is:

        $$H=-\vec{\mu}\cdot\vec{B}=\frac{e}{m_e}\vec{S}\cdot\vec{B}=\frac{eB_o}{m_e}S_z=\frac{\hbar\omega_o}{2}\left( \begin{matrix} 1 & 0\\ 0 &-1\end{matrix} \right)$$

      \item We can find eigenstates:

        $$H\ket{+}=\frac{\hbar}{2}\omega_o\ket{+}$$
        $$H\ket{-}=-\frac{\hbar}{2}\omega_o\ket{-}$$

      \item For a quantum state with initial state:

        $$\ket{\psi(0)}=\ket{+}$$

      \item We find time-evolution:

        $$\ket{\psi(t)}=e^{-i\omega_o t/2}\ket{+}$$

      \item Next, let us consider a more general state:

        $$\ket{\psi(0)}=\ket{+}_n=\cos\left( \frac{\theta}{2} \right)\ket{+}+\sin\left( \frac{\theta}{2} \right)e^{i\phi}\ket{-}$$

        \begin{itemize}

          \item Time-dependence may be expressed as:

            $$\ket{\psi(0)}=e^{-i\omega_ot/2}\cos\left( \frac{\theta}{2} \right)\ket{+}+e^{t_i\omega_ot/2}\sin\left( \frac{\theta}{2} \right)e^{i\phi}\ket{-}$$
            $$\ket{\psi(0)}=e^{-i\omega_ot/2}\left[\cos\left( \frac{\theta}{2} \right)\ket{+}+\sin\left( \frac{\theta}{2} \right)e^{i(\omega_ot+\phi)}\ket{-}\right]$$

        \end{itemize}

      \item We may use the time-dependence to find the probability of being in the up state:

        $$P_+=|\bra{+}\ket{\psi(t)}|^2=\cos^2(\theta/2)$$

        \begin{itemize}

          \item In this case, the probability is time independent since the $z$ component commutes with the Hamiltonian

          \item The angle $\theta$ the spin vector $\hat{n}$ makes with the $z$-axis does not change

        \end{itemize}

      \item Similarly, we may find:

        $$P_{+x}=|\,_x\bra{+}\ket{\psi(t)}|^2=\frac{1}{2}\left[ 1+\sin(\theta)\cos(\phi+\omega_o t) \right]$$

        \begin{itemize}

          \item We may see that this is time dependent since $H$ and $S_x$ do not commute

        \end{itemize}

      \item To see spin precession more clearly, we will compute the expectation values for each spin component

        $$<S_z>=\bra{\psi(t)|S_z}\ket{\psi(t)}=\frac{\hbar}{2}\cos(\theta)$$
        $$<S_y>=\bra{\psi(t)|S_y}\ket{\psi(t)}=\frac{\hbar}{2}\sin(\theta)\sin(\phi+\omega_o t)$$
        $$<S_x>=\bra{\psi(t)|S_x}\ket{\psi(t)}=\frac{\hbar}{2}\sin(\theta)\cos(\phi+\omega_o t)$$

        \begin{itemize}

          \item As such, we see that $<S_x>$ and $<S_y>$ oscillate in time and $<\vec{S}>$ precesses around the magnetic field with the angular frequency $\omega_o$, which is known as the Larmor frequency and precession, respectively

          \item The equivalence of the classical Larmor precession and the expectation value of the quantum spin vector is an example of Ehrenfest's theorem, which states that quantum expectation values obey classical laws

        \end{itemize}

      \item If we start a system in a spin $\ket{+}$ state and find the probability it has later evolved into a spin $\ket{-}$ state, we call this a spin flip

        $$P_{+\to-}=|\bra{-}\ket{\psi(t)}|^2=\sin^2(\theta)\sin^2\left( \frac{E_+-E_-}{2\hbar}t \right)=\frac{\omega_1^2}{\omega_o^2+\omega_o^2}\sin^2\left( \frac{\sqrt{\omega_o^2+\omega_1^2}}{2}t \right)$$

        \begin{itemize}

          \item This is known as Rabi's formula

        \end{itemize}

    \end{itemize}

\end{itemize}

\end{document}

