%%%%%%%%%%%%%%%%%%%%%%%%%%%%%%%%%%%%%%%%%%%%%%%%%%%%%%%%%%%%%%%%%%%%%%%%%%%%%%%%%%%%%%%%%%%%%%%%%%%%%%%%%%%%%%%%%%%%%%%%%%%%%%%%%%%%%%%%%%%%%%%%%%%%%%%%%%%%%%%%%%%
% Written By Michael Brodskiy
% Class: Quantum Mechanics
% Professor: G. Fiete
%%%%%%%%%%%%%%%%%%%%%%%%%%%%%%%%%%%%%%%%%%%%%%%%%%%%%%%%%%%%%%%%%%%%%%%%%%%%%%%%%%%%%%%%%%%%%%%%%%%%%%%%%%%%%%%%%%%%%%%%%%%%%%%%%%%%%%%%%%%%%%%%%%%%%%%%%%%%%%%%%%%

\documentclass[12pt]{article} 
\usepackage{alphalph}
\usepackage[utf8]{inputenc}
\usepackage[russian,english]{babel}
\usepackage{titling}
\usepackage{amsmath}
\usepackage{float}
\usepackage{graphicx}
\usepackage{enumitem}
\usepackage{amssymb}
\usepackage[super]{nth}
\usepackage{everysel}
\usepackage{ragged2e}
\usepackage{geometry}
\usepackage{multicol}
\usepackage{fancyhdr}
\usepackage{cancel}
\usepackage{siunitx}
\usepackage{physics}
\usepackage{tikz}
\usepackage{mathdots}
\usepackage{yhmath}
\usepackage{cancel}
\usepackage{color}
\usepackage{array}
\usepackage{multirow}
\usepackage{gensymb}
\usepackage{tabularx}
\usepackage{extarrows}
\usepackage{booktabs}
\usepackage{lastpage}
\usetikzlibrary{fadings}
\usetikzlibrary{patterns}
\usetikzlibrary{shadows.blur}
\usetikzlibrary{shapes}

\geometry{top=1.0in,bottom=1.0in,left=1.0in,right=1.0in}
\newcommand{\subtitle}[1]{%
  \posttitle{%
    \par\end{center}
    \begin{center}\large#1\end{center}
    \vskip0.5em}%

}
\usepackage{hyperref}
\hypersetup{
colorlinks=true,
linkcolor=blue,
filecolor=magenta,      
urlcolor=blue,
citecolor=blue,
}


\title{Lecture 4}
\date{\today}
\author{Michael Brodskiy\\ \small Professor: G. Fiete}

\begin{document}

\maketitle

\begin{itemize}

  \item Wave Equation for Unidimensional Particle

    \begin{itemize}

      \item In classical mechanics, we have the energy problem, with $E$ as the total energy, $T$ as the kinetic energy, and $V$ as the potential energy:

        $$E=T+V=\frac{p^2}{2m}+V(x)$$

      \item In quantum mechanics, we may define the Hamiltonian as:

        $$\hat{H}=\frac{\hat{p}}{2m}+V(\hat{x})$$

        \begin{itemize}

          \item Where $\hat{p}$ is the momentum operator and $\hat{x}$ is the position

        \end{itemize}

      \item From here, the time-independent Schr\"odinger equation may be written as:

        $$\hat{H}\phi_E(x)=E\phi_E(x)$$

        \begin{itemize}

          \item Where $\phi_E(x)$ represents the wave function/eigenfunction

        \end{itemize}

      \item We may continue to get:

        $$\hat{p}=-i\hbar\frac{d}{dx},\quad \hat{x}=x$$

        \begin{itemize}

          \item This can be used to obtain:

            $$\hat{p}\phi(x)=-i\hbar\frac{d}{dx}\phi(x),\quad \hat{x}\phi(x)=x\phi(x)$$

        \end{itemize}

      \item From here, we get:

        $$\hat{H}\psi(x)=-\frac{\hbar^2}{2m}\frac{d^2}{dx^s}\psi(x)+V(x)\psi(x)=E\psi(x)$$

      \item The above is the wave equation we need to solve. The wave function can generically be written as:

        $$\psi(x)=\sum_{n=0}^d\psi_n\phi_{E_n}(x)$$

        \begin{itemize}

          \item Note that $\psi_n$ is a scalar coefficient (projections of $\psi$ along the $n$-th direction) and $\phi_{E_n}(x)$ represents basis functions

          \item Also, note that, from previous lessons, we may recall that the probability of finding the system in a particular eigenstate is:

            $$P_{E_n}=|\psi_n|^2$$

        \end{itemize}

      \item In Dirac notation:

        $$\ket{\psi}=\left( \begin{matrix} \bra{E_1}\ket{\psi}\\ \bra{E_2}\ket{\psi}\\\vdots\\\bra{E_n}\ket{\psi} \end{matrix} \right)$$

        \begin{itemize}

          \item And also:

            $$\bra{\psi}=\left( \bra{E_1}\ket{\psi}^*\quad\bra{E_2}\ket{\psi}^*\quad\cdots\quad\bra{E_n}\ket{\psi}^* \right)$$

        \end{itemize}

      \item Change of Basis

        \begin{itemize}

          \item Changing basis to a position representation allows us to obtain the probability of finding the particle at $x$ as:

            $$P_x=|\psi(x)|^2$$

          \item This means that $|\psi(x)|^2$ is now a probability density such that:

            $$\int_{-\infty}^{\infty}P(x)\,dx=\int_{-\infty}^{\infty}|\psi(x)|^2\,dx=1$$

          \item Furthermore, we may find the probability that the particle is in a certain range as:

            $$P[a\leq x\leq b]=\int_{a}^{b}P(x)\,dx=\int_{a}^{b}|\psi(x)|^2\,dx$$

          \item In summary, we determine:

            $$\bra{x}\ket{\psi}=\psi(x)$$
            $$\bra{\psi}\ket{x}=\psi^*(x)$$
            $$\hat{A}=\hat{A}(x)$$

        \end{itemize}

    \end{itemize}

  \item Quantized Energies and Spectroscopy

    \begin{itemize}

      \item Spectroscopy is an experimental technique for measuring the energy fingerprint of a system

      \item Historically, hydrogen played an important role in the development of this technique

      \item Downward transitions give rise to emission spectra

      \item Upward transitions give rise to absorption spectra

      \item $E_i+E_j$, there is a possible spectral line with photon energy $E_i-E_j$, with photon frequency $f_{ij}$ and wavelength $\lambda_{ij}$:

        $$f_{ij}=\frac{\omega_{ij}}{2\pi}=\frac{E_i-E_j}{h}$$
        $$\lambda_{ij}=\frac{c}{f_{ij}}=\frac{hc}{E_i-E_j}$$

        \begin{itemize}

          \item Assuming $E_i-E_j>0$

        \end{itemize}

    \end{itemize}

  \item Infinite Square Well

    \begin{itemize}

      \item We want to solve:

        $$\left( -\frac{\hbar^2}{2m}\frac{d^2}{dx^2}+V(x) \right)\phi_E(x)=E\phi_E(x)$$

      \item The solutions of the this depend critically on the functional dependence of $V(x)$

      \item We create a variable $k^2$ such that:

        $$k^2=\frac{-2mE}{\hbar^2}$$

      \item This gives us:

        $$\frac{d^2}{dx^2}\phi_E(x)=k^2\phi_E(x)$$

      \item There are two possible forms of the solution:

        $$\phi_E(x)=Ae^{ikx}+Be^{-ikx}$$
        $$\phi_E(x)=A\sin(kx)+B\cos(kx)$$

      \item Applying boundary conditions, we obtain:

        $$k_n=\frac{n\pi}{L}$$

      \item From this, we may determine:

        $$E_n=\frac{n^2\hbar^2}{2mL^2},\quad n=1,2,3\cdots$$

      \item The general form of the wave function may be written:

        $$\phi_n(x)=\sqrt{\frac{2}{L}}\sin\left( \frac{n\pi x}{L} \right)$$

      \item We can compute expectation values as:

        $$<\hat{x}>=\bra{E_n|\hat{x}}\ket{E_n}=\int_{-\infty}^{\infty}\phi^{*}(x)x\phi_n(x)\,dx$$

        This gives us:

        $$\int_{-\infty}^{\infty}x|\phi_n(x)|^2\,dx=\frac{L}{2}$$

    \end{itemize}

  \item Finite Square Well

    \begin{itemize}

      \item In a finite well, we have potential energy defined by:

        $$V(x)=\left\{\begin{array}{ll} V_o,& x<-a\\0,&-a<x<a\\ V_o,&x>a\end{array}$$

        \item This gives us:

          $$\left( -\frac{\hbar^2}{2m}\fraC{d^2}{dx^2} \right)\phi_E(x)=E\phi_E(x)\quad\text{ (inside box)}$$
          $$\left( -\frac{\hbar^2}{2m}\fraC{d^2}{dx^2} +V_o\right)\phi_E(x)=E\phi_E(x)\quad\text{ (outside box)}$$

        \item We know that:

          $$k=\sqrt{\frac{2mE}{\hbar^2}}\quad\text{ (inside)}$$
          $$q=\sqrt{\frac{2m(V_o-E)}{\hbar^2}}\quad\text{ (outside)}, 0<E<V_o$$

        \item We may find the solutions inside and outside of the box (respectively) as:

          $$\phi_E(x)=e^{-ikx}\text{ or }\phi_E(x)=e^{-kx}\quad\text{ (inside)}$$
          $$\phi_E(x)=Ae^{qx}+Be^{-qx}\quad\text{ (outside)}$$

        \item Thus, we may write:

          $$\phi_E(x)=\left\{\begin{array}{ll} Ae^{qx}+Be^{-qx},& x<-a\\C\sin(kx)+D\cos(kx),&-a<x<a\\ Fe^{qx}+Ge^{-qx},&x>a\end{array}$$

        \item Two boundary conditions:

          \begin{enumerate}

            \item $\phi_E(x)$ is continuous

            \item $d\phi_E(x)/dx$ is continuous (unless the potential is infinite)

          \end{enumerate}

        \item Since our problem is symmetric about the origin, we have even and odd solutions:

          $$\phi_{even}(x)=\left\{\begin{array}{ll} Ae^{qx},& x<-a\\D\cos(kx),&-a<x<a\\ Ae^{-qx},&x>a\end{array}$$
          $$\phi_{odd}(x)=\left\{\begin{array}{ll} Ae^{qx},& x<-a\\C\sin(kx),&-a<x<a\\ -Ae^{-qx},&x>a\end{array}$$

    \end{itemize}

  \item General Remarks

    \begin{itemize}

      \item When $E>V$, the curvature of the wave function has the opposite sign

      \item When $E<V$, the curvature has the same sign as the wave function

      \item $k=\sqrt{2m(E-V)}$ and $\lambda=2\pi/k$, so:

        $$\lambda=\frac{h}{\sqrt{2m(E-V)}}\propto \frac{1}{\sqrt{T}}$$

        \begin{itemize}

          \item We may say that the wavelength is inversely proportional to the square root of the kinetic energy

        \end{itemize}

      \item In the forbidden region, the decay constant is:

        $$e^{-qx}\to q=\frac{\sqrt{2m(V-E)}}{\hbar}$$

    \end{itemize}

  \item Inversion Symmetry and Parity

    $$\hat{H}(x)=\hat{H}(-x)$$

    \begin{itemize}

      \item The Hamiltonian is invariant under parity

        $$[\hat{\text{Parity}},\hat{H}]=0$$

      \item Energy eigenstates are also eigenstates of the parity operator

        $$\hat{\text{Parity}}\phi_n(x)=+\phi_n(-x)\quad \text{ even parity}$$
        $$\hat{\text{Parity}}\phi_n(x)=-\phi_n(-x)\quad \text{ odd parity}$$

    \end{itemize}

  \item Superposition States and Time-Dependence

    \begin{itemize}

      \item We start from:

        $$H\ket{\psi}=i\hbar\frac{d}{dt}\ket{\psi}$$

        \begin{itemize}

          \item With:

            $$\ket{\psi}=\sum_n c_Ne^{-iE_nt/\hbar}\ket{E_n}$$

          \item In general, an initial state will be of the form:

            $$\ket{\psi(0)}=\sum_nc_n\ket{E_n}$$
            
            \begin{itemize}

              \item Where $c_n=\bra{E_n}\ket{\psi(0)}=\displaystyle \int \phi^*_n(x)\psi(x,t=0)\,dx$

            \end{itemize}
          
          \item We may observe that, with this, we get:

            $$<\hat{x}>=\frac{L}{2}\left[ 1-\frac{32}{9\pi^2}\cos\left( \frac{3\pi^2\hbar}{2mL^2}t \right) \right]\quad \text{ (note the oscillation with Bohr frequency)}$$
            $$<\hat{p}>=\frac{8}{3}\frac{\hbar}{2}\sin\left( \frac{3\pi^2\hbar}{2mL^2}t \right)$$

          \item This is another example of Ehrenfest's theorem: quantum expectation values obey classical laws

        \end{itemize}

    \end{itemize}

\end{itemize}

\end{document}

