%%%%%%%%%%%%%%%%%%%%%%%%%%%%%%%%%%%%%%%%%%%%%%%%%%%%%%%%%%%%%%%%%%%%%%%%%%%%%%%%%%%%%%%%%%%%%%%%%%%%%%%%%%%%%%%%%%%%%%%%%%%%%%%%%%%%%%%%%%%%%%%%%%%%%%%%%%%%%%%%%%%
% Written By Michael Brodskiy
% Class: Quantum Mechanics
% Professor: G. Fiete
%%%%%%%%%%%%%%%%%%%%%%%%%%%%%%%%%%%%%%%%%%%%%%%%%%%%%%%%%%%%%%%%%%%%%%%%%%%%%%%%%%%%%%%%%%%%%%%%%%%%%%%%%%%%%%%%%%%%%%%%%%%%%%%%%%%%%%%%%%%%%%%%%%%%%%%%%%%%%%%%%%%

\documentclass[12pt]{article} 
\usepackage{alphalph}
\usepackage[utf8]{inputenc}
\usepackage[russian,english]{babel}
\usepackage{titling}
\usepackage{amsmath}
\usepackage{float}
\usepackage{graphicx}
\usepackage{enumitem}
\usepackage{amssymb}
\usepackage[super]{nth}
\usepackage{everysel}
\usepackage{ragged2e}
\usepackage{geometry}
\usepackage{multicol}
\usepackage{fancyhdr}
\usepackage{cancel}
\usepackage{siunitx}
\usepackage{physics}
\usepackage{tikz}
\usepackage{mathdots}
\usepackage{yhmath}
\usepackage{cancel}
\usepackage{color}
\usepackage{array}
\usepackage{multirow}
\usepackage{gensymb}
\usepackage{tabularx}
\usepackage{extarrows}
\usepackage{booktabs}
\usepackage{lastpage}
\usetikzlibrary{fadings}
\usetikzlibrary{patterns}
\usetikzlibrary{shadows.blur}
\usetikzlibrary{shapes}

\geometry{top=1.0in,bottom=1.0in,left=1.0in,right=1.0in}
\newcommand{\subtitle}[1]{%
  \posttitle{%
    \par\end{center}
    \begin{center}\large#1\end{center}
    \vskip0.5em}%

}
\usepackage{hyperref}
\hypersetup{
colorlinks=true,
linkcolor=blue,
filecolor=magenta,      
urlcolor=blue,
citecolor=blue,
}


\title{Lecture 4}
\date{\today}
\author{Michael Brodskiy\\ \small Professor: G. Fiete}

\begin{document}

\maketitle

\begin{itemize}

  \item Wave Equation for Unidimensional Particle

    \begin{itemize}

      \item In classical mechanics, we have the energy problem, with $E$ as the total energy, $T$ as the kinetic energy, and $V$ as the potential energy:

        $$E=T+V=\frac{p^2}{2m}+V(x)$$

      \item In quantum mechanics, we may define the Hamiltonian as:

        $$\hat{H}=\frac{\hat{p}}{2m}+V(\hat{x})$$

        \begin{itemize}

          \item Where $\hat{p}$ is the momentum operator and $\hat{x}$ is the position

        \end{itemize}

      \item From here, the time-independent Schr\"odinger equation may be written as:

        $$\hat{H}\phi_E(x)=E\phi_E(x)$$

        \begin{itemize}

          \item Where $\phi_E(x)$ represents the wave function/eigenfunction

        \end{itemize}

      \item We may continue to get:

        $$\hat{p}=-i\hbar\frac{d}{dx},\quad \hat{x}=x$$

        \begin{itemize}

          \item This can be used to obtain:

            $$\hat{p}\phi(x)=-i\hbar\frac{d}{dx}\phi(x),\quad \hat{x}\phi(x)=x\phi(x)$$

        \end{itemize}

      \item From here, we get:

        $$\hat{H}\psi(x)=-\frac{\hbar^2}{2m}\frac{d^2}{dx^s}\psi(x)+V(x)\psi(x)=E\psi(x)$$

      \item The above is the wave equation we need to solve. The wave function can generically be written as:

        $$\psi(x)=\sum_{n=0}^d\psi_n\phi_{E_n}(x)$$

        \begin{itemize}

          \item Note that $\psi_n$ is a scalar coefficient (projections of $\psi$ along the $n$-th direction) and $\phi_{E_n}(x)$ represents basis functions

          \item Also, note that, from previous lessons, we may recall that the probability of finding the system in a particular eigenstate is:

            $$P_{E_n}=|\psi_n|^2$$

        \end{itemize}

      \item In Dirac notation:

        $$\ket{\psi}=\left( \begin{matrix} \bra{E_1}\ket{\psi}\\ \bra{E_2}\ket{\psi}\\\vdots\\\bra{E_n}\ket{\psi} \end{matrix} \right)$$

        \begin{itemize}

          \item And also:

            $$\bra{\psi}=\left( \bra{E_1}\ket{\psi}^*\quad\bra{E_2}\ket{\psi}^*\quad\cdots\quad\bra{E_n}\ket{\psi}^* \right)$$

        \end{itemize}

      \item Change of Basis

        \begin{itemize}

          \item Changing basis to a position representation allows us to obtain the probability of finding the particle at $x$ as:

            $$P_x=|\psi(x)|^2$$

          \item This means that $|\psi(x)|^2$ is now a probability density such that:

            $$\int_{-\infty}^{\infty}P(x)\,dx=\int_{-\infty}^{\infty}|\psi(x)|^2\,dx=1$$

          \item Furthermore, we may find the probability that the particle is in a certain range as:

            $$P[a\leq x\leq b]=\int_{a}^{b}P(x)\,dx=\int_{a}^{b}|\psi(x)|^2\,dx$$

          \item In summary, we determine:

            $$\bra{x}\ket{\psi}=\psi(x)$$
            $$\bra{\psi}\ket{x}=\psi^*(x)$$
            $$\hat{A}=\hat{A}(x)$$

        \end{itemize}

    \end{itemize}

  \item Quantized Energies and Spectroscopy

    \begin{itemize}

      \item Spectroscopy is an experimental technique for measuring the energy fingerprint of a system

      \item Historically, hydrogen played an important role in the development of this technique

      \item Downward transitions give rise to emission spectra

      \item Upward transitions give rise to absorption spectra

      \item $E_i+E_j$, there is a possible spectral line with photon energy $E_i-E_j$, with photon frequency $f_{ij}$ and wavelength $\lambda_{ij}$:

        $$f_{ij}=\frac{\omega_{ij}}{2\pi}=\frac{E_i-E_j}{h}$$
        $$\lambda_{ij}=\frac{c}{f_{ij}}=\frac{hc}{E_i-E_j}$$

        \begin{itemize}

          \item Assuming $E_i-E_j>0$

        \end{itemize}

    \end{itemize}

  \item Infinite Square Well

    \begin{itemize}

      \item We want to solve:

        $$\left( -\frac{\hbar^2}{2m}\frac{d^2}{dx^2}+V(x) \right)\phi_E(x)=E\phi_E(x)$$

      \item The solutions of the this depend critically on the functional dependence of $V(x)$

      \item We create a variable $k^2$ such that:

        $$k^2=\frac{-2mE}{\hbar^2}$$

      \item This gives us:

        $$\frac{d^2}{dx^2}\phi_E(x)=k^2\phi_E(x)$$

      \item There are two possible forms of the solution:

        $$\phi_E(x)=Ae^{ikx}+Be^{-ikx}$$
        $$\phi_E(x)=A\sin(kx)+B\cos(kx)$$

      \item Applying boundary conditions, we obtain:

        $$k_n=\frac{n\pi}{L}$$

      \item From this, we may determine:

        $$E_n=\frac{n^2\hbar^2}{2mL^2},\quad n=1,2,3\cdots$$

      \item The general form of the wave function may be written:

        $$\phi_n(x)=\sqrt{\frac{2}{L}}\sin\left( \frac{n\pi x}{L} \right)$$

      \item We can compute expectation values as:

        $$<\hat{x}>=\bra{E_n|\hat{x}}\ket{E_n}=\int_{-\infty}^{\infty}\phi^{*}(x)x\phi_n(x)\,dx$$

        This gives us:

        $$\int_{-\infty}^{\infty}x|\phi_n(x)|^2\,dx=\frac{L}{2}$$

    \end{itemize}

  \item Finite Square Well

    \begin{itemize}

      \item In a finite well, we have potential energy defined by:

        $$V(x)=\left\{\begin{array}{ll} V_o,& x<-a\\0,&-a<x<a\\ V_o,&x>a\end{array}$$

        \item This gives us:

          $$\left( -\frac{\hbar^2}{2m}\fraC{d^2}{dx^2} \right)\phi_E(x)=E\phi_E(x)\quad\text{ (inside box)}$$
          $$\left( -\frac{\hbar^2}{2m}\fraC{d^2}{dx^2} +V_o\right)\phi_E(x)=E\phi_E(x)\quad\text{ (outside box)}$$

        \item We know that:

          $$k=\sqrt{\frac{2mE}{\hbar^2}}\quad\text{ (inside)}$$
          $$q=\sqrt{\frac{2m(V_o-E)}{\hbar^2}}\quad\text{ (outside)}, 0<E<V_o$$

        \item We may find the solutions inside and outside of the box (respectively) as:

          $$\phi_E(x)=e^{-ikx}\text{ or }\phi_E(x)=e^{-kx}\quad\text{ (inside)}$$
          $$\phi_E(x)=Ae^{qx}+Be^{-qx}\quad\text{ (outside)}$$

        \item Thus, we may write:

          $$\phi_E(x)=\left\{\begin{array}{ll} Ae^{qx}+Be^{-qx},& x<-a\\C\sin(kx)+D\cos(kx),&-a<x<a\\ Fe^{qx}+Ge^{-qx},&x>a\end{array}$$

        \item Two boundary conditions:

          \begin{enumerate}

            \item $\phi_E(x)$ is continuous

            \item $d\phi_E(x)/dx$ is continuous (unless the potential is infinite)

          \end{enumerate}

        \item Since our problem is symmetric about the origin, we have even and odd solutions:

          $$\phi_{even}(x)=\left\{\begin{array}{ll} Ae^{qx},& x<-a\\D\cos(kx),&-a<x<a\\ Ae^{-qx},&x>a\end{array}$$
          $$\phi_{odd}(x)=\left\{\begin{array}{ll} Ae^{qx},& x<-a\\C\sin(kx),&-a<x<a\\ -Ae^{-qx},&x>a\end{array}$$

    \end{itemize}

  \item General Remarks

    \begin{itemize}

      \item When $E>V$, the curvature of the wave function has the opposite sign

      \item When $E<V$, the curvature has the same sign as the wave function

      \item $k=\sqrt{2m(E-V)}$ and $\lambda=2\pi/k$, so:

        $$\lambda=\frac{h}{\sqrt{2m(E-V)}}\propto \frac{1}{\sqrt{T}}$$

        \begin{itemize}

          \item We may say that the wavelength is inversely proportional to the square root of the kinetic energy

        \end{itemize}

      \item In the forbidden region, the decay constant is:

        $$e^{-qx}\to q=\frac{\sqrt{2m(V-E)}}{\hbar}$$

    \end{itemize}

  \item Inversion Symmetry and Parity

    $$\hat{H}(x)=\hat{H}(-x)$$

    \begin{itemize}

      \item The Hamiltonian is invariant under parity

        $$[\hat{\text{Parity}},\hat{H}]=0$$

      \item Energy eigenstates are also eigenstates of the parity operator

        $$\hat{\text{Parity}}\phi_n(x)=+\phi_n(-x)\quad \text{ even parity}$$
        $$\hat{\text{Parity}}\phi_n(x)=-\phi_n(-x)\quad \text{ odd parity}$$

    \end{itemize}

  \item Superposition States and Time-Dependence

    \begin{itemize}

      \item We start from:

        $$H\ket{\psi}=i\hbar\frac{d}{dt}\ket{\psi}$$

        \begin{itemize}

          \item With:

            $$\ket{\psi}=\sum_n c_Ne^{-iE_nt/\hbar}\ket{E_n}$$

          \item In general, an initial state will be of the form:

            $$\ket{\psi(0)}=\sum_nc_n\ket{E_n}$$
            
            \begin{itemize}

              \item Where $c_n=\bra{E_n}\ket{\psi(0)}=\displaystyle \int \phi^*_n(x)\psi(x,t=0)\,dx$

            \end{itemize}
          
          \item We may observe that, with this, we get:

            $$<\hat{x}>=\frac{L}{2}\left[ 1-\frac{32}{9\pi^2}\cos\left( \frac{3\pi^2\hbar}{2mL^2}t \right) \right]\quad \text{ (note the oscillation with Bohr frequency)}$$
            $$<\hat{p}>=\frac{8}{3}\frac{\hbar}{2}\sin\left( \frac{3\pi^2\hbar}{2mL^2}t \right)$$

          \item This is another example of Ehrenfest's theorem: quantum expectation values obey classical laws

        \end{itemize}

    \end{itemize}

  \item Unbound States

    \begin{itemize}

      \item Occur when $E>V_o$

      \item The simplest unbound state is a free particle where $V(x)=0$ everywhere:

        $$\frac{d^2}{dx^2}[\phi_E(x)]=-k^2\phi_E(x)$$

      \item The solutions are:

        $$\phi_E(x)=Ae^{ikx}+Be^{-ikx}$$

        \begin{itemize}

          \item Where there is no condition to ``fit'' a certain number of wavelengths in a region

        \end{itemize}

      \item To understand the free particle function, consider:

        $$\psi_E(x,t)=\phi_E(x)e^{-iEt/\hbar}$$
        $$\psi_E(x,t)=\left[  Ae^{ikx}+Be^{-ikx}\right]e^{-iEt/\hbar}$$
        $$\psi_E(x,t)=Ae^{i(kx-\omega t)/\hbar}+Be^{-i(kx-\omega t)/\hbar}$$

        \begin{itemize}

          \item Observe that this function has a form familiar to classical physics: $f(x\pm vt)$

          \item This form retains its shape as it moves with speed $v=\omega/k$, or the phase velocity
            
          \item The energy eigenstate has a right and left-moving part

        \end{itemize}

      \item It is convenient to use wave vector eigenstates:

        $$\phi_k(x)=Ae^{ikx}$$
        $$\phi_E(x)=\phi_{+k}(x)+\phi_{-k}(x)$$

    \end{itemize}

  \item Momentum Eigenstates

    \begin{itemize}

      \item We may find that:

        $$p=\hbar k$$

      \item As well as:

        $$\phi_p(x)=Ae^{ipx/\hbar}$$

        \begin{itemize}

          \item Where $x$ is a variable, and $p$ is a particular momentum:

            $$p=\hbar k=\frac{h}{2\pi}\cdot\frac{2\pi}{\lambda}=\frac{h}{\lambda}$$

          \item This represents the de-Broglie wavelength:

            $$\lambda_{dB}=\frac{h}{p}$$
            $$E=\frac{p^2}{2m}\Rightarrow [H,p]=0$$

            \begin{itemize}

              \item This implies that energy and momentum share eigenstates

              \item In our case, we have a degeneracy, where $\ket{p}$ and $\ket{-p}$ have the same energy

              \item Returning to the issue of the phase velocity, we can write:

                $$\psi_p(x,t)=Ae^{i\frac{p}{\hbar}\left( x-\frac{pt}{2m} \right)}$$

              \item This implies that the wave's speed is $v=\frac{p}{2m}$, despite the classical speed being $p/m$

              \item We should properly describe the velocity of a particle by the ``group velocity of a wave function''

            \end{itemize}

          \item A more serious problem with the momentum eigenstates appears when one examines the probability density of the state

            $$P(x)=|\phi_p(x)|^2=|A|^2\quad\text{ (constant)}$$

            \begin{itemize}

              \item This leads to two issues:

                \begin{enumerate}

                  \item The particle is equally likely to be at any location

                  \item The wave function can not be normalized over all space

                \end{enumerate}

              \item Solution: construct wave packets built from a superposition of momentum eigenstates

              \item All of the discrete basis states we have encountered satisfy these conditions:

                $$\bra{a_i}\ket{a_{j\neq i}}=0\quad\text{ (orthogonality)}$$
                $$\bra{a_i}\ket{a_{i}}=1\quad\text{ (normality)}$$
                $$\sum \ket{a_i}\bra{a_i}=\mathbb{1}\quad\text{ (completeness)}$$
                $$\bra{a_i}\ket{a_j}=\delta_{ij}\quad\text{ (orthonormality)}$$

              \item To extend orthonormality to a continuous basis, one promotes a Kronecker delta function to a Dirac delta function:

                $$\bra{p''}\ket{p'}=\delta(p''-p')$$

              \item Where:

                $$\int_{-\infty}^{\infty} \phi^*(x)_{p''}(x)\phi_{p'}(x)\,dx=\delta(p''-p')$$

              \item For this to be the case, we must take:

                $$A=\frac{1}{\sqrt{2\pi \hbar}}\Rightarrow \phi_p(x)=\frac{1}{\sqrt{2\pi \hbar}}e^{ipx/\hbar}$$

              \item This makes:

                $$|\phi_p(x)|^2\,dx$$

              \item have units of length divided by $\hbar$, which means so does the Dirac delta function

            \end{itemize}

          \item For momentum eigenstates, the completeness relation is:

            $$\int_{-\infty}^{\infty}\ket{p}\bra{p}\,dp=\mathbb{1}$$

          \item We may obtain:

            $$\psi(x)=\int_{-\infty}^{\infty}\phi_p(x)\psi(p)\,dp$$
            $$\psi(x)=\frac{1}{\sqrt{2\pi\hbar}}\int_{-\infty}^{\infty}\phi(p)e^{ipx/\hbar}\,dp$$

            \begin{itemize}

              \item This clearly shows that $\psi(x)$ is the Fourier transform of $\phi(p)$. Thus, $\phi(p)$ is the Fourier transform of the position wave function:

                $$\psi(x)=\frac{1}{\sqrt{2\pi\hbar}}\int_{-\infty}^{\infty}\psi(x)e^{-ipx/\hbar}\,dx$$

            \end{itemize}

        \end{itemize}

    \end{itemize}

  \item Wave Packets

    \begin{itemize}

      \item We may create discrete wave packets:

        $$p_o-\delta p, p_o, p_o+\delta p$$

        \begin{itemize}

          \item This leads to something local in space

        \end{itemize}

      \item We begin by writing:

        $$\psi(x,0)=\sum_{j}c_j\phi_{p_j}(x)$$

        \begin{itemize}

          \item This expands to:

            $$\psi(x,0)=\frac{1}{\sqrt{2\pi\hbar}}\left[ \frac{1}{2}e^{i(p_o-\delta p}x/\hbar+e^{ip_ox/\hbar}+\frac{1}{2}e^{i(p_o+\delta p)x/\hbar} \right]$$
          \item We may get the time-dependent form as:

            $$\psi(x,t)=\frac{1}{\sqrt{2\pi\hbar}}e^{ip_o\left( x-\frac{p_ot}{2m\hbar} \right)}\left[ 1+\cos\left( \frac{\delta p x}{\hbar}-\frac{p_o\delta pt}{m\hbar} \right) \right]$$

            \begin{itemize}

              \item We observe that:

                $$v_{phase}=\frac{p_o}{2m}$$

              \item The exponential term in the time-dependent form is referred to as the ``carrier wave''

              \item The sinusoidal term is referred to as the ``envelope''

              \item Note that, in the envelope, the group velocity becomes:

                $$v_{group}=\frac{p_o}{m}$$

              \item The sinusoid modulates the carrier wave

            \end{itemize}

        \end{itemize}

      \item We can make a continuous superposition of momentum states as:

        $$\psi(x,0)=\int_{-\infty}^{\infty}\phi(p)\frac{1}{\sqrt{2\phi\hbar}}e^{ipx/\hbar}\,dp$$

      \item We then get:

        $$\psi(x,t)=\int_{-\infty}^{\infty}\phi(p)\frac{1}{\sqrt{2\phi\hbar}}e^{ipx/\hbar}e^{-ip^2t/2m\hbar}\,dp$$
        $$\phi(p)=\int_{-\infty}^{\infty}\psi(x,t)\frac{1}{\sqrt{2\phi\hbar}}e^{-ipx/\hbar}\,dp$$

      \item We choose $\phi(p)$ to be a Gaussian centered at $p_o$:

        $$\phi(p)=\left( \frac{1}{2\pi\beta} \right)^{1/2}e^{-(p-p_o)^2/4\beta^2}$$

        \begin{itemize}

          \item With $\langle p\rangle=p_o$, and $\Delta p=\beta$

        \end{itemize}

      \item Using this, along with the commutator, we obtain:

        $$\Delta x\Delta p=\frac{\hbar}{2}\sqrt{1+\left( \frac{2\beta^2t}{m\hbar}\right)^2}$$

        \begin{itemize}

          \item Note that at $t=0$, we simply have the uncertainty as $\hbar/2$

        \end{itemize}

    \end{itemize}

  \item Unbound States and Scattering

    \begin{itemize}

      \item Consider the following potential:

        $$V(x)=\left\{\begin{array}{ll} 0,&x<-a\\-V_o,& -a<x<a\\0,&x>a\end{array}$$

      \item Bound states occurs at $E<0$ and scattering states occur at $E>0$

      \item This gives us:

        $$\phi_E(x)=\left\{\begin{array}{ll} Ae^{ik_1x}+Be^{-ik_1x},&x<-a\\Ce^{ik_2x}+De^{-ik_2x},& -a<x<a\\Fe^{ik_1x}+Ge^{-ik_1x},&x>a\end{array}$$

    \end{itemize}

  \item Transmission Coefficient ($T$)

    \begin{itemize}

      \item We may write the coefficient as:

        $$T=\Big|\frac{F}{A}\Big|^2=\frac{1}{1+\frac{(k_1-k_2)^2}{4k_1^2k_2^2}\sin^2(2k_2a)}$$

    \end{itemize}

  \item Reflection Coefficient ($R$)

    \begin{itemize}

      \item Similarly, we may write the coefficient as:

        $$R=\Big|\frac{B}{A}\Big|^2=\frac{1}{1+\frac{4k_1^2k_2^2}{(k_1^2-k_2^2)\sin^2(2k_2a)}}$$

      \item The two coefficients are related to each other in that:

        $$T+R=1$$

    \end{itemize}

  \item Angular Momentum

    \begin{itemize}

      \item One of the most important problems in the history of quantum mechanics: the hydrogen atom

      \item Bound states of a proton and an electron

      \item We have 2 particles and 3 dimensions, so there are new technical challenges

      \item There are a few problems we must solve, beginning with the hamiltonian of the system:

        $$H_{sys}\psi_{sys}(\vec{R},\vec{F})=E_{sys}\psi_{sys}(\vec{R},\vec{r})$$

        \begin{itemize}

          \item This means we need to solve for the wave function of the system:

            $$\psi_{sys}(\vec{R},\vec{r})=\psi_{cm}(\vec{R})\psi_{rel}(\vec{r})$$

          \item We need to break these into:

            $$H_{cm}\psi_{cm}(\vec{R})=E_{cm}\psi_{cm}(\vec{R})\quad\text{ and }\qua H_{rel}\psi_{rel}(\vec{r})=E_{rel}\psi_{rel}(\vec{r})$$

          \item Although the former is fairly simple since $\psi_{cm}(x,y,z)$ is in terms of $x,y,z$, the latter becomes trickier, since:

            $$\psi_{rel}(r,\theta,\phi)=R(r)\Theta(\theta)\Phi(\phi)$$

          \item Thus, the solution of the problem will take the form:

            $$\psi_{sys}(\vec{R},\vec{r})=\psi_{cm}(x,y,z)\psi_{rel}(r,\theta,\phi)$$

          \item We write the Hamiltonian as:

            $$H_{sys}=\frac{p_1^2}{2m_1}+\frac{p_2^2}{2m_2}+V(\vec{r}_1,\vec{r}_2)$$

          \item The Coulomb potential between the electron and proton is central:

            $$V(\vec{r}_1,\vec{r}_2)=V(|\vec{r}_1-\vec{r}_2|)$$

          \item We decompose the motion into center of mass motion, and motion about the center of mass (relative motion):

              $$\text{Center of Mass:}\quad \vec{R}=\frac{m_1\vec{r}_1+m_2\vec{r}_2}{m_1+m_2}$$
              $$\vec{r}=\vec{r}_2-\vec{r}_1$$
              $$\vec{p}=\vec{p}_1+\vec{p}_2$$
              $$\text{Relative Momentum: }\quad \vec{p}_{rel}=\frac{m_1\vec{p}_2-m_2\vec{p}_1}{m_1+m_2}$$
              $$\frac{\vec{p}_{rel}}{\mu}=\frac{\vec{p}_2}{m_2}-\frac{\vec{p}_1}{m_1}$$
              $$\frac{1}{\mu}=\frac{1}{m_1}+\frac{1}{m_2}\quad\text{ or }\mu=\frac{m_1m_2}{m_1+m_2}$$

            \item Using the above, as well as $M=m_1+m_2$, we write:

              $$H_{sys}=\frac{\vec{p}^2}{2M}+\left( \frac{\vec{p}_{rel}^2}{2\mu}+V(r) \right)$$

            \item Using the independence of the coordinates, we may obtain the eigenvalue equations as:

              $$\frac{\vec{p}^2}{2M}\psi_{cm}(\vec{R})=E_{cm}\psi_{cm}(\vec{R})$$
              $$\left(\frac{\vec{p}_{rel}^2}{2m}+V(r)\right)\psi_{rel}(\vec{r})=E_{rel}\psi_{rel}(\vec{r})$$

            \item We can thus find the energy of the center of mass as:

              $$E_{cm}=\frac{1}{2M}\left( p_x^2+p_y^2+p_z^2 \right)$$

              \begin{itemize}

                \item With wave function:

                  $$\psi_{cm}(x,y,z)=\frac{1}{(2\pi\hbar)^{3/2}}e^{i(p_xx+p_yy+p_zz)/\hbar}$$

              \end{itemize}

            \item We now move to the more complex relative motion case (eigenvalue equation in spherical coordinates):

              $$H=\frac{p^2}{2\mu}+V(r)$$
              $$\left( -\frac{\hbar^2}{2\mu}\nabla^2+V(r) \right)\psi(\vec{r})=E\psi(\vec{r})$$

        \end{itemize}

      \item Returning to Angular Momentum, we recall that, classically:

        $$\vec{\tau}=\vec{r}\times\vec{p}$$

      \item For a central source, we know:

        $$\frac{d\vec{\tau}}{dt}=0$$

        \begin{itemize}

          \item This implies $\vec{L}$ is a constant

          \item Quantum Mechanically, we may write the components of the angular momentum as:

            $$L_x=yp_z-zp_y=-i\hbar\left( y\frac{\partial}{\partial z}-z\frac{\partial}{\partial y} \right)$$
            $$L_y=zp_x-xp_z=-i\hbar\left( z\frac{\partial}{\partial x}-x\frac{\partial}{\partial z} \right)$$
            $$L_z=xp_y-yp_x=-i\hbar\left( x\frac{\partial}{\partial y}-y\frac{\partial}{\partial x} \right)$$

            \begin{itemize}

              \item We may recall:

                $$[x,p_x]=i\hbar$$
                $$[x,p_y]=[x,p_z]=0$$

            \end{itemize}

          \item We can then return to find:

            $$[L_x,L_y]=i\hbar L_z$$

          \item And ultimatelly we enter this into the above equations to write:

            $$[L_x,L_y]=i\hbar L_z$$
            $$[L_y,L_z]=i\hbar L_x$$
            $$[L_z,L_x]=i\hbar L_y$$

          \item Furthermore, we may consider:

            $$[\vec{L}^2,L_x]=[\vec{L}^2,L_y]=[\vec{L}^2,L_z]=0$$

          \item Therefore, we may complete the analogy:

            $$\vec{s}^2\ket{sm_s}=s(s+1)\hbar^2\ket{sm_s}\to\vec{L}^2\ket{lm_l}=l(l+1)\hbar^2\ket{lm_l}$$
            $$L_z\ket{lm_l}=m_l\hbar\ket{lm_l}$$

            \begin{itemize}

              \item Where $l$ is the orbital angular momentum quantum number, and $m_l$ is the orbital magnetic quantum number

              \item It is important to note a crucial difference:

                $$s=0,1/2,1,3/2,\cdots$$
                $$l=0,1,2,3,\cdots$$
            
              \item For $m_l$, we are constrained by $2l+1$:

                $$m_l=-l,-l+1,\cdots l-1, l$$

            \end{itemize}

        \end{itemize}

    \end{itemize}

  \item Orbital Angular Momentum:

    \begin{itemize}

      \item Angular momentum along the $z$ axis can be written in operator form using:

        $$L_z=-i\hbar\frac{d}{d\phi}$$

      \item Since the following equal zero, we can tell that they are simultaneour eigenstates of $3$ operators:

        $$[H,\vec{L}^2]=[H,L_z]=0$$

      \item Separation of Variables: Spherical Coordinates

        \begin{itemize}

          \item Assume $\psi(r,\theta,\phi)=R(r)Y(\theta,\phi)$

          \item We substitute to get:

            $$-\frac{\hbar^2}{2\mu}\left[ Y(\theta,\phi)\frac{1}{r^2}\frac{d}{dr}\left( r^2\frac{d}{dr}R(r) \right)-\frac{R(r)}{\hbar^2r^2}\vec{L}^2Y(\theta,\phi) \right]+V(r)R(r)Y(\theta,\phi)=ER(r)Y(\theta,\phi)$$

          \item We divide both sides by $R(r)Y(\theta,\phi)$ to get:

            $$-\frac{\hbar^2}{2\mu}\left[ \frac{1}{r^2R(r)}\frac{d}{dr}\left( r^2\frac{d}{dr}R(r) \right)-\frac{1}{\hbar^2r^2}\vec{L}^2 \right]+V(r)=E$$

          \item We isolate all coordinate dependence to one side of the equation:

          $$\frac{1}{R(r)}\frac{d}{dr}\left( r^2\frac{d}{dr}R(r) \right)-\frac{2\mu}{\hbar^2}(E-V(r))r^2 \right]=\frac{1}{\hbar^2}\frac{1}{Y(\theta,\phi)}\vec{L}\psi(\theta,\phi)$$

          \item We may continue separating variables:

            $$Y(\theta,\phi)=\Theta(\theta)\Phi(\phi)$$

          \item Thus, we get:

            $$\left[ \frac{1}{\sin(\theta)}\frac{d}{d\theta}\left\sin(\theta)\frac{d}{d\theta}\right)-\frac{B}{\sin^2(\theta)} \right]\Theta(\theta)=-A\Theta(\theta)$$
            $$\frac{d^2\Phi(\phi)}{d\phi^2}=-B\Phi(\phi)$$

          \item Solving lets us find:

            $$\Phi_m(\phi)=\bra{\phi}\ket{m}=\frac{1}{\sqrt{2\pi}}e^{im\phi}$$

        \end{itemize}

    \end{itemize}

  \item Motion on a Sphere

    \begin{itemize}

      \item Goal: Combine solutions of $\Theta(\theta)$ and $\Phi_m(\phi)$ to find $Y(\theta,\phi)$, the so-called spherical harmonics

      \item We can confine a particle of mass $\mu$ to a spherical plane of radius $r_o$
        
      \item Using the confined $r_o$ radius, we may state that $V(r_o)=0$, thus giving us:

        $$\frac{\vec{L}^2}{2I} Y(\theta,\phi)=E_{sphere}Y(\theta,\phi)$$

        \begin{itemize}

          \item Note that $I$ is the moment of intertia, or $\mu r_o^2$

          \item This agrees with the separation form we obtained:

            $$A=\frac{2I}{\hbar^2}E_{sphere}=l(l+1)$$

        \end{itemize}

      \item Legendre Polynomials:

        $$P_l(z)=\frac{1}{2^ll!}\frac{d^l}{dz^l}(z^2-1)^l$$

      \item Lagrange Polynomials:

        $$P_l^m(z)=P_l^{-m}(z)=(1-z^l)^{m/2}\frac{d^m}{dz^m}P_l(z)$$

    \end{itemize}

\end{itemize}

\end{document}

