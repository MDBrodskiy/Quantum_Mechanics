%%%%%%%%%%%%%%%%%%%%%%%%%%%%%%%%%%%%%%%%%%%%%%%%%%%%%%%%%%%%%%%%%%%%%%%%%%%%%%%%%%%%%%%%%%%%%%%%%%%%%%%%%%%%%%%%%%%%%%%%%%%%%%%%%%%%%%%%%%%%%%%%%%%%%%%%%%%%%%%%%%%
% Written By Michael Brodskiy
% Class: Quantum Mechanics
% Professor: G. Fiete
%%%%%%%%%%%%%%%%%%%%%%%%%%%%%%%%%%%%%%%%%%%%%%%%%%%%%%%%%%%%%%%%%%%%%%%%%%%%%%%%%%%%%%%%%%%%%%%%%%%%%%%%%%%%%%%%%%%%%%%%%%%%%%%%%%%%%%%%%%%%%%%%%%%%%%%%%%%%%%%%%%%

\include{Includes.tex}

\title{Lecture 4}
\date{\today}
\author{Michael Brodskiy\\ \small Professor: G. Fiete}

\begin{document}

\maketitle

\begin{itemize}

  \item Wave Equation for Unidimensional Particle

    \begin{itemize}

      \item In classical mechanics, we have the energy problem, with $E$ as the total energy, $T$ as the kinetic energy, and $V$ as the potential energy:

        $$E=T+V=\frac{p^2}{2m}+V(x)$$

      \item In quantum mechanics, we may define the Hamiltonian as:

        $$\hat{H}=\frac{\hat{p}}{2m}+V(\hat{x})$$

        \begin{itemize}

          \item Where $\hat{p}$ is the momentum operator and $\hat{x}$ is the position

        \end{itemize}

      \item From here, the time-independent Schr\"odinger equation may be written as:

        $$\hat{H}\phi_E(x)=E\phi_E(x)$$

        \begin{itemize}

          \item Where $\phi_E(x)$ represents the wave function/eigenfunction

        \end{itemize}

      \item We may continue to get:

        $$\hat{p}=-i\hbar\frac{d}{dx},\quad \hat{x}=x$$

        \begin{itemize}

          \item This can be used to obtain:

            $$\hat{p}\phi(x)=-i\hbar\frac{d}{dx}\phi(x),\quad \hat{x}\phi(x)=x\phi(x)$$

        \end{itemize}

      \item From here, we get:

        $$\hat{H}\psi(x)=-\frac{\hbar^2}{2m}\frac{d^2}{dx^s}\psi(x)+V(x)\psi(x)=E\psi(x)$$

      \item The above is the wave equation we need to solve. The wave function can generically be written as:

        $$\psi(x)=\sum_{n=0}^d\psi_n\phi_{E_n}(x)$$

        \begin{itemize}

          \item Note that $\psi_n$ is a scalar coefficient (projections of $\psi$ along the $n$-th direction) and $\phi_{E_n}(x)$ represents basis functions

          \item Also, note that, from previous lessons, we may recall that the probability of finding the system in a particular eigenstate is:

            $$P_{E_n}=|\psi_n|^2$$

        \end{itemize}

      \item In Dirac notation:

        $$\ket{\psi}=\left( \begin{matrix} \bra{E_1}\ket{\psi}\\ \bra{E_2}\ket{\psi}\\\vdots\\\bra{E_n}\ket{\psi} \end{matrix} \right)$$

        \begin{itemize}

          \item And also:

            $$\bra{\psi}=\left( \bra{E_1}\ket{\psi}^*\quad\bra{E_2}\ket{\psi}^*\quad\cdots\quad\bra{E_n}\ket{\psi}^* \right)$$

        \end{itemize}

      \item Change of Basis

        \begin{itemize}

          \item Changing basis to a position representation allows us to obtain the probability of finding the particle at $x$ as:

            $$P_x=|\psi(x)|^2$$

          \item This means that $|\psi(x)|^2$ is now a probability density such that:

            $$\int_{-\infty}^{\infty}P(x)\,dx=\int_{-\infty}^{\infty}|\psi(x)|^2\,dx=1$$

          \item Furthermore, we may find the probability that the particle is in a certain range as:

            $$P[a\leq x\leq b]=\int_{a}^{b}P(x)\,dx=\int_{a}^{b}|\psi(x)|^2\,dx$$

          \item In summary, we determine:

            $$\bra{x}\ket{\psi}=\psi(x)$$
            $$\bra{\psi}\ket{x}=\psi^*(x)$$
            $$\hat{A}=\hat{A}(x)$$

        \end{itemize}

    \end{itemize}

\end{itemize}

\end{document}

