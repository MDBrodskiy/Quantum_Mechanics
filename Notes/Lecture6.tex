%%%%%%%%%%%%%%%%%%%%%%%%%%%%%%%%%%%%%%%%%%%%%%%%%%%%%%%%%%%%%%%%%%%%%%%%%%%%%%%%%%%%%%%%%%%%%%%%%%%%%%%%%%%%%%%%%%%%%%%%%%%%%%%%%%%%%%%%%%%%%%%%%%%%%%%%%%%%%%%%%%%
% Written By Michael Brodskiy
% Class: Quantum Mechanics
% Professor: G. Fiete
%%%%%%%%%%%%%%%%%%%%%%%%%%%%%%%%%%%%%%%%%%%%%%%%%%%%%%%%%%%%%%%%%%%%%%%%%%%%%%%%%%%%%%%%%%%%%%%%%%%%%%%%%%%%%%%%%%%%%%%%%%%%%%%%%%%%%%%%%%%%%%%%%%%%%%%%%%%%%%%%%%%

\include{Includes.tex}

\title{Lecture 6}
\date{\today}
\author{Michael Brodskiy\\ \small Professor: G. Fiete}

\begin{document}

\maketitle

\begin{itemize}

  \item Harmonic Oscillator

    \begin{itemize}

      \item Classical

        $$F=-kx$$
        $$V(x)=\frac{1}{2}kx^2$$
        $$F=-\frac{dV}{dx}$$

      \item Quantum

        $$E=\frac{p^2}{2m}+\frac{1}{2}kx^2$$
        $$H=\frac{\hat{p}^2}{2m}+\frac{1}{2}m\omega^2\hat{x}^2$$
        $$\omega=\sqrt{\frac{k}{m}}$$

        \begin{itemize}

          \item Energy eigenvalues can be found as:

            $$E_n=\hbar\omega(n+1/2)$$

          \item We may find:

            $$H=\hbar\omega(aa^{\dagger}+1/2)$$

          \item And from here, we find:

            $$[a,a^{\dagger}]=1$$

            \begin{itemize}

              \item This indicates that the operators $a$ and $a^{\dagger}$ raise and lower the energy eigenstates

              \item We can write this as:

                $$a\ket{E}\propto \ket{E-\hbar\omega}$$
                $$a^{\dagger}\ket{E}\propto \ket{E+\hbar\omega}$$

              \item These are called ``ladder operators''

              \item Note that there is an asymmetry in the ladder, since $aa^{\dagger}\neq a^{\dagger}a$

              \item Since there is a lowest energy state in the harmonic oscillator well, states can not be lowered in energy indefinitely, such that:

                $$a\ket{E_{lowest}}=0$$

              \item This is called the ladder termination condition

                $$H\ket{E_{lowest}}=\hbar\omega(aa^{\dagger}+1/2)\ket{E_{lowest}}=\frac{\hbar\omega}{2}\ket{E_{lowest}}$$

              \item Thus, we may conclude that the lowest energy is $\hbar\omega/2$

              \item Since this is finite, we say the quantum mechanical ground state has a zero-point energy of $\hbar\omega/2$

            \end{itemize}

        \end{itemize}

    \end{itemize}

  \item Excited States

    \begin{itemize}

      \item We may obtain:
        
        $$\ket{n}=\frac{(a^{\dagger})^n}{\sqrt{n!}}\ket{0}$$
        $$\phi_n(x)=\frac{1}{\sqrt{n!}}\left[ \sqrt{\frac{m\omega}{2\hbar}}\left( x-\frac{\hbar}{m\omega}\frac{d}{dx} \right)^n \right]\phi_0(x)$$

    \end{itemize}

  \item Dirac Notation

    $$\ket{n}=\ket{\phi_n}=\ket{E_n}=\ket{(n+1/2)\hbar\omega}$$

    \begin{itemize}

      \item With $\phi_n(x)=\bra{x}\ket{n}$

      \item Since $\braket{n}=1$, and $\displaystyle \int_{-\infty}^{\infty} \ket{x}\bra{x}\,dx=\mathbb{1}$

        $$1=\bra{n}\int_{-\infty}^{\infty}\ket{x}\bra{x}\,dx\ket{n}=\int_{-\infty}^{\infty} \phi_n^*(x)\pxi_n(x)\,dx$$

      \item By orthonormality, we have:

        $$\delta_{mn}=\bra{m}\ket{n}=\int_{-\infty}^{\infty} \phi_m^*(x)\phi_n(x)\,dx$$

      \item Since the Hermitian operator states are eigenstates of the Hamiltonian, they form a complete set of states, such that:

        $$\sum_{n=0}^{\infty} \ket{n}\bra{n}=\mathbb{1}$$

      \item A general state $\ket{\psi}$ can be written as:

        $$\ket{\psi}=\sum_{n=0}^{\infty} (\underbrace{\bra{n}\ket{\psi}}_{c_n})\ket{n}$$

      \item We know:

        $$c_n=\int_{-\infty}^{\infty}\phi_n^*(x)\psi(x)\,dx$$

      \item The probability of the state $\ket{\psi}$ having energy $E_n$ is:

        $$P_{E_n}=|\bra{n}\ket{\psi}|^2=|c_n|^2$$

    \end{itemize}

\end{itemize}

\end{document}

